\documentclass[../notes.tex]{subfiles}

\begin{document}

\chapter{Introduction to Group Theory} \label{ch:grp}
\epigraph{The philosophy is that any time the reader sees a definition or a theorem about such an object, they should test it against the prototypical example.}
{---Evan Chen, \cite{napkin}}
In this chapter, we discuss groups as our most basic algebraic object. As such, we will discuss groups as one discusses anything in algebra: we examine how groups fit inside each other, how they map to each other, how they decompose, and so on. Because groups are more abstract than real analysis, we are sure to ground our study with many examples.

% \subfile{mods}
\subfile{introgroups}
\subfile{coset}
\subfile{homs}

\end{document}