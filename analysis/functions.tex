%Read the slack post (link below) regarding syntax and formatting before you start writing lecture notes.
% Post: https://musa-2021.slack.com/archives/C01DGR645SL/p1609187728029500


\documentclass[../notes.tex]{subfiles}
\begin{document}

\stepcounter{week}
\section{Week \theweek: Continuous Functions}

One of the most fundamental ideas you learned in calculus was that of a continuous function. Your instructor might have said "a funciton is continuous if you can draw it without picking your pencil up from the paper," or something like that. This week we'll give a rigorous definition for continuity, but before that we should look at limits of functions whose domain is $\RR$.

\subsection{Limits of Functions}

Here is our definition of limits of real functions.
\begin{definition}[limit]
    Let $f\colon \RR \to \RR$ be a real-valued function, $a\in \RR$, $L\in \RR$. Then the \dfn{limit}\textit{of $f(x)$ as $x$ approaches $a$} is $L$ if and only if, for all $\varepsilon>0$, there exists $\delta>0$ such that $|f(x)-L| <\varepsilon$ when $ 0 < |x-a| < \delta$. We write this as $\lim_{x\to a}f(x)=L$ or $f(x)\to L$ as $x\to a$.
\end{definition}
Compare this definition to that of convergent sequences: ``$n\geq N$'' is replaced with ``$0 < |x-a|\leq \delta$,'' and ``$|x_n-x| < \varepsilon$'' with ``$|f(x)-L| < \varepsilon$.'' We are purposefully excluding the case $|x-a|=0$ because the behavior of the function at $a$ should not change its limit. Here is a rough image.
\begin{center}
    \begin{asy}
        unitsize(1cm);
        draw((-2,0)--(2,0),red);
        draw((0,-2)--(0,2),dotted);
        fill(circle((0,0),0.1),white);
        draw(circle((0,0),0.1));
        dot((0,1),red);
        label("$f(x)$", (1,0), N, red);
    \end{asy}
\end{center}
Intuitively, we would like $f(x)\to0$ as $x\to0$ even though $f(0)=1$. We explain this more rigorously in the following example.
\begin{example} \label{ex:rem-discont}
    Define $f\colon\RR\to\RR$ by
    \[ f(x) \coloneqq \begin{cases}
        1 & \text{if }x=0, \\
        0 & \text{if }x\neq 0.
    \end{cases}\]
    From what you've learned in calculus, $\lim_{x\to 0}f(0)$ should equal $0$. Indeed, suppose $\varepsilon>0$. Then for any $\delta >0$, $0 < |x-0| < \delta$ implies $x\neq 0$, so $|f(x)-0| = |0-0| = 0< \varepsilon$, and therefore $\lim_{x\to 0}f(x) =0$. Suppose, however, our definition included the case $|x-a|=0$. Then $|f(0)-0|=1 > \frac{1}{2}$, so $\lim_{x\to 0}f(x) \neq 0$ for this alternative definition.
\end{example}
%TODO include picture
%TODO show limit is unique
Anyway, as in \Cref{prop:seq-limit-uniq}, we ought to show that the limit defined above is unique when it exists.
\begin{proposition}
    Let $f\colon\RR\to\RR$ be a real-valued function, and choose $a\in\RR$. If real numbers $L_1$ and $L_2$ satisfy $f(x)\to L_1$ and $f(x)\to L_2$ as $x\to a$, then $L_1=L_2$.
\end{proposition}
\begin{proof}
    The proof is analogous to \Cref{prop:seq-limit-uniq}. The main claim is that $|L_1-L_2|<\varepsilon$ for any $\varepsilon>0$. Indeed, by definition of our convergence, for any $\varepsilon>0$, we are promised $\delta_1>0$ and $\delta_2>0$ such that each $i\in\{1,2\}$ has
    \[\left|f(x)-L_i\right|<\frac\varepsilon2\]
    for $0<|x-a|<\delta_i$. We can use the above inequality to measure the distance between $L_1$ and $L_2$ as follows: for any $x$ satisfying $0<|x-a|<\min\{\delta_1,\delta_2\}$, we use the triangle inequality of \Cref{prop:real-metric} to see
    \[\left|L_1-L_2\right|\le\left|f(x)-L_1\right|+\left|f(x)-L_2\right|<\frac\varepsilon2+\frac\varepsilon2=\varepsilon.\]
    This completes the proof of the main claim.

    We now finish the proof. Suppose for the sake of contradiction that $L_1\ne L_2$. Then $\varepsilon\coloneqq\left|L_1-L_2\right|>0$, which is a contradiction to the main claim because the main claim requires $\left|L_1-L_2\right|<\varepsilon$. This completes the proof.
\end{proof}
%TODO maybe change this? 
\begin{remark}
   Some important examples of real-valued functions are only defined on subsets of $\RR$, like $f(x)=\sqrt{x}$. Let $A\subset \RR$ and $f\colon A\to \RR$. Then the above definition still works for $\lim_{x\to a}f(x)$, except we only require $|f(x)-L|\leq \varepsilon$ for $\{x : 0 < |x-a| < \delta\}\cap A$. 
\end{remark}
Here is a small army of examples.
\begin{example} \label{ex:const-limit}
    Let $a\in \RR$, and $f$ be the constant function $f(x) \coloneqq a$. We'll show that
    \[ \lim_{x\to a}f(x) =a.\]
\end{example}
\begin{proof}
    Let $\varepsilon>0$. Then $|f(x)-a| =|a-a|=0$ for all $x\in \RR$. So for any choice of $\delta>0$, we see $|f(x)-a| = 0 <\varepsilon$ whenever $0< |x-a| <\delta$.
\end{proof}

% Now for a more interesting example: 

\begin{example} \label{ex:affine-limit}
    Consider the linear function $f(x)  \coloneqq 2x+1$. We'll show that
    \[ \lim_{x\to 2}f(x) =5.\]
\end{example}
\begin{proof}
    To begin, let's simplify the distance between $f(x)$ and $5$, writing
    \begin{flalign*}
        |f(x)-5| &= |2x+1-5| \\
        &= |2x-4| \\
        &= 2|x-2|.
    \end{flalign*}
    Let $\varepsilon>0$. Then for $\delta\coloneqq\frac{\varepsilon}{2}$, we see $|f(x)-5| < \varepsilon$ whenever $|x-2|<\delta$.
\end{proof}

% We'll finish with a slightly tougher example. 

\begin{example}
    Let $f(x)=x^2$. We'll show that
    \[ \lim_{x\to 1}f(x) = 1.\]
\end{example}
\begin{proof}
    Notice that
    \begin{flalign*}
       |f(x)=1| &= |x^2-1|
       = |x-1|\cdot|x+1|.
    \end{flalign*}
    Now let $\delta=\min\{\frac{\varepsilon}{2}, 1\}$. If $|x-1|<\delta$, then the triangle inequality implies $|x+1| <2 $ (you should check this). Thus
    \begin{flalign*}
        |f(x)-1| &= |x-1|\cdot|x+1| \\
        &\leq \delta |x+1| \\
        &\leq \frac{\varepsilon}{2}\cdot2 \\
        &= \varepsilon,
    \end{flalign*}
    which completes the proof.
\end{proof}
\begin{remark}
    %TODO is this necessary?
    Notice that we took $\delta=\min\{\frac{\varepsilon}{2}, 1\}$ since we wanted the benefits of both $\delta =\frac{\varepsilon}{2}$ and $\delta=1$. This was similar to how we chose $N=\max\{N_1,N_2\}$ to get the benefits of both $n\geq N_1$ and $n\geq N_2$.
\end{remark}
The ideas behind sequential convergence and limits of functions seem very similar, and in fact they have a special relation: 

%TODO FINISH
\begin{prop}
\label{prop:seqs-and-limits}
    If $f\colon\RR\to \RR$, the following statements are equivalent:
    \begin{enumerate}[label=(\roman*)]
        \item $\lim_{x\to a}f(x) = L$
        \item If $\{x_n\}$ is a convergent sequence such that $x_n\neq a$ for all $n$ but $x_n\to a$, then $\{f(x_n)\}$ converges to $L$. 
    \end{enumerate}
\end{prop}
%TODO explain what's happening here
\begin{proof}
    We show the implications separately.
    \begin{itemize}
        \item We show (i) implies (ii). In this case, the continuity of the function translates nicely into the continuity of the function.
        
        Let $\varepsilon>0$. Because $\lim_{x\to a}f(x)=L$, there is $\delta>0$ such that $|f(x)-L| < \varepsilon$ whenever $|x-a|<\delta$. Because $x_n \to a$, there is $N$ such that $|x_n-a| < \delta$ whenever $n\geq N$. Thus, $|f(x_n)-L| < \varepsilon$ whenever $n\geq N$, and therefore $\lim_{n\to \infty} f(x_n) = L$.
        \item We show that if (i) is false, then (ii) is false. The main point is to use the failure of $f(x)\to L$ in order to build a sequence $\{x_n\}$ breaking (ii).
        
        If $\lim_{x\to a}f(x) \neq L$, then there is some $\varepsilon>0$ such that no $\delta > 0$ exists satisfying the definition of continuity. We would like to convert this condition into something workable, so we note that breaking $f(x)\to L$ means that for any $\delta > 0$, there is a real number $z_\delta$ such that
        \[0 < |z_\delta - a | < \delta\qquad\text{but}\qquad |f(z_\delta) - L| > \varepsilon.\]
        We now convert the real numbers $z_\delta$ into the desired sequence. For each $n\in \NN$ let $x_n \coloneqq z_{1/n}$. This gives us a sequence $\{x_n\}$ such that $0 < |x_n- a| < \frac{1}{n}$ and $|f(x_n) - L|  >\varepsilon$ for each $n$.
        
        To complete the proof, we would like to show that $x_n\ne a$ for each $n$ while $x_n\to a$, and $\{f(x_n)\}$ does not converge to $L$. We check these one at a time. Because $|x_n-a|>0$ for each $n$, we see $x_n\ne a$. To see $x_n\to a$, we recall that $|x_n-a|<\frac1n$ for each $n$, so any $\varepsilon_0>0$ can set $N\coloneqq1/\varepsilon_0$ so that
        \[|x_n-a|<\frac1n\le\frac1N=\varepsilon_0\]
        for any $n>N$.

        Lastly, we want to check that $\{f(x_n)\}$ does not converge to $L$. Well, suppose for the sake of contradiction that $f(x_n)\to L$ as $n\to\infty$. Then taking $\varepsilon>0$ as constructed above, $f(x_n)\to L$ requires some $N$ such that $\left|f(x_n)-L\right|<\varepsilon$ for any $n\ge N$. But this directly contradicts the construction of $\varepsilon$, completing the contradiction.
    \end{itemize}
    The above implications complete the proof.
\end{proof}


\Cref{prop:seqs-and-limits} means that we can use sequential convergence and limits interchangeably and apply results from last week to limits of functions!

%TODO finish proof
\begin{proposition}
\label{prop:lim-is-linear}
    Suppose $f,g\colon \RR\to \RR$ are functions such that such that $\lim_{x\to a}f(x)$ and $\lim_{x\to a}g(x)$ exist.
    \begin{listalph}
       \item Let $h(x)=f(x)+g(x)$. Then \[\lim_{x\to a}h(x) = \lim_{x\to a}f(x)+\lim_{x\to a}g(x)\] 
       \item Let $\lambda \in \RR$ and $h(x) = \lambda f(x)$. Then 
       \[ \lim_{x\to a}h(x) = \lambda \lim_{x\to a}f(x)\]
       \item Let $h(x)=f(x)g(x)$. Then 
       \[ \lim_{x\to a}h(x) = \left(\lim_{x\to a}f(x)\right)\left(\lim_{x\to a}g(x)\right)\]
    \end{listalph}
\end{proposition}
\begin{proof}
    We will show (a) and leave the remaining parts to \Cref{exe:finish-lim-is-linear}.
    
    This is basically a corollary of \Cref{prop:conv-is-linear} and \Cref{prop:seqs-and-limits}: let $\{x_n\}$ be any sequence converging to $a$ such that $x_n\neq a$. Then, by \Cref{prop:seqs-and-limits}, $\lim_{n\to \infty}f(x_n) = \lim_{x\to a}f(x)$ and $\lim_{n\to \infty}g(x_n) = \lim_{x\to a}g(x)$. Thus, by \Cref{prop:conv-is-linear},
    \begin{flalign*}
        \lim_{n\to \infty} h(x_n) &= \lim_{n\to \infty} f(x_n) + \lim_{n\to \infty} g(x_n)
        = \lim_{x\to a}f(x) + \lim_{x\to a}g(x).
    \end{flalign*}
    Since this is true for all sequences $\{x_n\}$ such that $x_n\to a$ and $x_n\ne a$ for each $n$, applying \Cref{prop:seqs-and-limits} again implies that
    \[\lim_{x\to a}h(x) = \lim_{x\to a}f(x)+\lim_{x\to a}g(x),\] 
    completing the proof of (a).
\end{proof}

\begin{exe} \label{exe:finish-lim-is-linear}
    Prove parts (b) and (c) of \Cref{prop:lim-is-linear}
\end{exe}

\subsection{Continuous Functions}

With an understanding of how to compute limits of functions, we are able to introduce the important notion of continuity.
%TODO include some transition 

\begin{definition}[continuous]
    Let $f\colon\RR \to \RR$ be a function. Then $f$ is \dfn{continuous}\textit{ at $a\in\RR$} if and only if
    \[\lim_{x\to a}f(x) = f(a).\]
    Further, $f$ is \dfn{continuous} if and only if $f$ is continuous at $a$ for all $a\in \RR$. 
\end{definition}
Here are some quick examples.
\begin{example}
    Fix $c\in \RR$ and consider the constant function $f(x)\coloneqq c$. We showed that $\lim_{x\to a}f(x)=c=f(a)$ in \Cref{ex:const-limit} for all $a\in \RR$, so $f$ is a continuous function. 
\end{example}
\begin{example}
    Consider the function $f(x)\coloneqq x$. If $x_n\to a$, then clearly $f(x_n)\to a$, so by \Cref{prop:seqs-and-limits} we have $\lim_{x\to a}f(x) = a = f(a)$. Thus, $f$ is continuous.
\end{example}
Because of \Cref{prop:lim-is-linear}, we immediately get the following facts.

\begin{proposition} \label{prop:linear-combo-cont}
    Let $f, g\colon\RR\to \RR$ be continuous functions (or continuous at $a\in \RR$) and $\lambda\in \RR$. Then $f+g$, $\lambda f$, and $f\cdot g$ are all continuous (respectively, continuous at $a\in \RR$). 
\end{proposition}
\begin{proof}
    Observe that the ``full'' continuity result follows directly from showing the continuity at $a\in\RR$ result. Explicitly, if we can show that $f+g$, $\lambda f$, and $f\cdot g$ are continuous $a\in\RR$ when $f$ and $g$ are continuous at $a\in\RR$, then when $f$ and $g$ are continuous at all $a\in\RR$, we see that $f+g$, $\lambda f$, and $f\cdot g$ will also be continuous at all $a\in\RR$.

    As such, we will content ourselves with focusing on a single $a\in\RR$. We will show that $f+g$ is continuous at $a\in\RR$ because the proofs for the other statements are analogous and thus left to \Cref{exe:complete-linear-combo-cont}. Set $h\coloneqq f+g$ for brevity. Now, by \Cref{prop:lim-is-linear}, we see that
    \[\lim_{x\to a}h(x)=\lim_{x\to a}f(x)+\lim_{x\to a}g(x)=f(a)+g(a)=h(a),\]
    which is what we wanted.
\end{proof}
\begin{exe} \label{exe:complete-linear-combo-cont}
    Complete the proof of \Cref{prop:linear-combo-cont} by showing that $\lambda f$ and $f\cdot g$ are continuous at $a\in\RR$ when $f$ and $g$ are continuous at $a\in\RR$.
\end{exe}

\begin{example}
    Let $m, b\in \RR$ and consider the affine-linear function $f(x)\coloneqq mx+b$. We showed $g(x)\coloneqq x$ is continuous, so $h(x)\coloneqq mx$ is continuous by \Cref{prop:lim-is-linear}. We also showed the constant function $k(x) = b$ is continuous, so \Cref{prop:lim-is-linear} implies that $f(x) = h(x) +  k(x)$ is continuous. Note that this result generalizes \Cref{ex:affine-limit}.
\end{example}

\begin{proposition}
    Let $f,g\colon\RR\to \RR$ be continuous functions. Then the composite $f\circ g$ is a continuous function. 
\end{proposition}
\begin{proof}
    We argue directly. Fix $a\in \RR$, and we want to show that $f(g(x))\to f(g(a))$ as $x\to a$. Well, fix any $\varepsilon>0$, and we want $\delta>0$ such that $\left|f(g(x))-f(g(a))\right|<\varepsilon$ for any $0<|x-a|<\delta$.
    
    To unwrap the $f$ from our desired conclusion, we use the continuity of $f$ to find some $\delta_1>0$ such that $|f(y)-f(g(a))|<\varepsilon$ for any $0<|y-g(a)|<\delta_1$. Thus,
    \[\left|f(g(x))-f(g(a))\right|<\varepsilon\qquad\text{if}\qquad0<\left|g(x)-g(a)\right|<\delta_1.\]
    As a technical point, note that the conclusion $\left|f(g(x))-f(g(a))\right|<\varepsilon$ remains true when $g(x)=g(a)$, so we could also write this as
    \[\left|f(g(x))-f(g(a))\right|<\varepsilon\qquad\text{if}\qquad\left|g(x)-g(a)\right|<\delta_1.\]
    From here, we use the continuity of $g$ to find $\delta>0$ such that $\left|g(x)-g(a)\right|<\delta_1$ for all $0<|x-a|<\delta$. Synthesizing, we see
    \[\left|f(g(x))-f(g(a))\right|<\varepsilon\qquad\text{if}\qquad\left|x-a\right|<\delta,\]
    which is what we wanted.
\end{proof}


%\begin{example}
%   $\sqrt{x}$ is continuous. 
%\end{example}


\subsection{Flavors of Discontinuity}
Next up, let's look at some examples of functions that are not continuous. 

\begin{definition}[discontinuity]
    If $f$ is not continuous at $x=a$, then $a$ is called a \dfn{discontinuity} point of $f$. 
\end{definition}

\begin{example} \label{ex:rem-discont-2}
    Consider the function from \Cref{ex:rem-discont}, defined as
    \[ f(x) \coloneqq \begin{cases}
        1 & \text{if }x=0, \\
        0 & \text{if }x\neq 0.
    \end{cases}\]
    We showed that $\lim_{x\to 0}f(x) =0$, so $f$ is not continuous at $a=0$. But because $\lim_{x\to a}f(x)=0=f(a)$ for $a\neq 0$, $f$ is continuous at $a\neq 0$. 
\end{example}
The condition of $f$ in \Cref{ex:rem-discont-2} is really not ``too serious'' in that $f$ is essentially continuous at $0$ except that the value of $f(0)$ did not quite align. We will give this kind of discontinuity a name.
\begin{definition}[removable discontinuity]
    If $\lim_{x\to a}f(x)$ exists but does not equal $f(a)$, we call $a$ a \dfn{removable discontinuity} of $f$. 
\end{definition}
In calculus you likely discussed limits ``from the left/right", which allow us to isolate the behavior of the function to the left/right of a point. More precisely, we have the following definition.

\begin{definition}[left/right limits]
    Let $f\colon\RR\to \RR$ be a function, and let $a,L\in \RR$ be real numbers. Then the \dfn{left limit} $\lim_{x\to a^-}f(x)=L$ if and only if for all $\varepsilon>0$, there is $\delta>0$ such that $a-\delta < x < a$ implies $|f(x)-L| < \varepsilon$. Similarly, the \dfn{right limit} $\lim_{x\to a^+}f(x)=L$ if and only if for all $\varepsilon>0$, there is $\delta>0$ such that $a < x < a+\delta$ implies $|f(x)-L| < \varepsilon$.
\end{definition}
Basically, we are only requiring $f(x)$ to be close to $L$ for $x$ to the left/right of $a$. 

\begin{exe}
    Consider the function 
    \[ f(x) \coloneqq \begin{cases}
        0 & x < 0 \\
        1 & x \geq 0
    \end{cases}\]
    Show that $\lim_{x\to 0^-}f(x) = 0$, while $\lim_{x\to 0^+}f(x) = 1$. 
\end{exe}

\begin{definition}
    If $\lim_{x\to a^-}f(x)$ and  $\lim_{x\to a^+}f(x)$ both exist but are different, we say $a$ is a \dfn{jump discontinuity} of $f$.
\end{definition}
We should really prove that jump discontinuities are indeed discontinuity points. This was likely presented to you as another definition of continuity in calculus class.

\begin{proposition}
    We have $\lim_{x\to a}f(x)=L$ if and only if $\lim_{x\to a^+}f(x)=lim_{x\to a^-}f(x)=L$.
\end{proposition}

\begin{proof}
    We show the implications separately.
    \begin{itemize}
        \item Suppose $\lim_{x\to a}f(x)=L$. Roughly speaking this is the ``stronger'' limit result, so left and right limits will follow somewhat directly.
        
        For any $\varepsilon>0$, choose $\delta>0$ such that $0 < |x-a| < \delta$ implies $|f(x)-L| < \varepsilon$. In particular, $a- \delta < x < a$ or $a < x < a + \delta$ implies $|f(x)-L| < \varepsilon$. Thus $\lim_{x\to a^-}f(x)=\lim_{x\to a^-}f(x)=L$.
        \item Suppose $\lim_{x\to a^-}=\lim_{x\to a^-}$. Roughly speaking, the point is to glue the left and right limits into the full limit $\lim_{x\to a}f(x)$.
        
        For each $\varepsilon>0$, find $\delta^-, \delta^+>0>0$ such that $|f(x)-L| < \varepsilon$ when $a-\delta^- < x < a$ or $a < x < a+\delta^+$. Then $|f(x)-L| < \varepsilon$ when $0 < |x-a| < \min{\delta^+, \delta^-}$, so  $\lim_{x\to a}f(x)=L$ follows.
        \qedhere
    \end{itemize}
\end{proof}

\subsection{Problems}

\begin{homework}
    Let $m,b\in \RR$ and $f(x)\coloneqq mx+b$. Prove that $\lim_{x\to a}f(x)=f(a)$ for all $a\in \RR$ {using only what we have covered prior to \Cref{prop:seqs-and-limits}}. 
\end{homework}

\begin{homework}
     Show that \Cref{prop:seqs-and-limits} is true if we replace ``$x_n\neq a$ for all $n\in \NN$'' in (ii) with  ``$x_n\neq a$ for all but finitely many $n\in \NN$.'' 
 \end{homework}


\begin{homework}
    A \dfn{polynomial function} is a function of the form $p(x) \coloneqq a_nx^n+\dots + a_1x+a_0$, where $a_i\in \RR$. Using \Cref{prop:lim-is-linear}, show that all polynomial functions are continuous. %challenge: show this directly using the definition of a limit
\end{homework}

\begin{homework}
    Show that the function $f\colon\RR_{\geq 0}\to \RR$ defined by $f(x)\coloneqq\sqrt x$ is continuous at all $x\in \RR_{\geq 0}$. You may find it easier to deal with continuity at $x=0$ separately.
\end{homework}

\begin{homework}
    Let $f, g:\RR\to \RR$ be continuous functions. For $a\in \RR$, show that the function 
    \[ h(x)\coloneqq\begin{cases}
        f(x) &x< a \\
        g(x) &x\geq a
    \end{cases}\]
    is continuous if and only if $f(a)=g(a)$.
\end{homework}

\begin{homework}
    Let $f\colon\RR\to \RR$ and $t\in \RR$. Show that the translated function $h_t\colon\RR\to\RR$ defined by $h_t(x)\coloneqq f(t+x)$ is continuous if and only if $f$ is continuous. 
\end{homework}

\begin{homework}
    For a function $f\colon\RR\to\RR$, define the function $|f|$ by $x\mapsto|f(x)|$.
    \begin{listalph}
        \item If $f\colon\RR\to\RR$ is a continuous function, show that $|f|$ is continuous.
        \item Find a discontinuous function $f\colon\RR\to\RR$ such that $|f|$ is continuous.
    \end{listalph}
\end{homework}

\begin{homework}
    Let $f_1, f_2, \dots, f_n \colon \RR\to \RR$ be a finite set of continuous functions. Show that 
    \[ M(x) \coloneqq \max\{f_1(x), \dots, f_n(x)\}\]
    is continuous.
\end{homework}

\begin{homework}
    Let $f(x)=\frac{1}{x}$ for $x\neq 0$, and $f(0)=0$.
    \begin{listalph}
        \item Show that $f$ is continuous at $x$ if $x\neq 0$
        \item Show that $f$ is not continuous at $0$, and that $f$ is {not} a removable discontinuity.
    \end{listalph}
\end{homework}


\end{document}