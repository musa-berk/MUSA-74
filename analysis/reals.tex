%Read the slack post (link below) regarding syntax and formatting before you start writing lecture notes.
% Post: https://musa-2021.slack.com/archives/C01DGR645SL/p1609187728029500


\documentclass[../notes.tex]{subfiles}
\begin{document}

\stepcounter{week}
\section{Week \theweek: The Real Numbers}

In the last two weeks we've rigorously defined limits and continuity, but we've only managed to work with relatively simple functions. A study of more interesting functions (like $\log(x), e^x, $ and $sin(x)$) requires further ideas, all of which rest on a good understanding of real numbers. This week we'll discuss properties of $\RR$, although we will assume the existence of $\RR$ and that it satisfies property \Cref{prop:sup-prop}.

\subsection{Supremum and Infimum}
Consider the sequence $\{x_n\}$ defined by $x_n\coloneqq\frac{1}{n}$ for all $n$. Then $\{x_n\}$ is ``monotonically decreasing,'' meaning $x_n \leq x_m$ if $n > m$. When does a monotonically decreasing sequence $\{x_n\}$ converge? Looking at a number line, it seems like it will happen as long as the sequence is bounded from below. Here is a visual example where the sequence is bounded from below.
\begin{center}
    \begin{asy}
        unitsize(1cm);
        draw((0,-1)--(0,2),Arrows);
        draw((0,0)--(5,0),dashed);
        for(int i = 1; i < 5; ++i)
        {
            //dot("$"+string(i)+"$",(i,0),S);
            dot("$x_"+string(i)+"$", (i,1/i), N);
        }
    \end{asy}
\end{center}
In fact, our conjecture is true, and we will prove it in \Cref{thm:mono-inc-converges}.
%TODO insert picture

\begin{definition}[bounded from below/above]
    Let $A$ be a set of real numbers. Then $A$ is \dfn{bounded from below} (respectively, \dfn{bounded from above}) if and only if there is a real number $r$ such that $r\leq a$ (respectively, $r\ge a$) for all $a\in A$. In this situation, $r$ is called a \dfn{lower bound} (respectively, \dfn{upper bound}) of $A$.
\end{definition}

\begin{example}
    \begin{listalph}
        \item The set $A\coloneqq\NN$ is bounded from below because $0$ is a lower bound. However, $A$ is not bounded from above because for any $r\in \RR$ we can set $n = |\ceil{r}| + 1 > r$. 
        \item The set $A\coloneqq\ZZ$ is neither bounded from below nor bounded from above.
        \item If $A$ is a finite set, $\max(A)$ is an upper bound, so $A$ bounded from above. Further, $\min(A)$ is a lower bound, so $A$ is bounded from below. 
    \end{listalph}
\end{example}
The same ideas suggest that a monotonically increasing sequence that is bounded from above will converge. To prove this, we could try setting $x=\max\{x_n : n\in \NN\}$ (the set of elements of the sequence). However, $\{x_n : n\in \NN\}$ is infinite, it may not have a maximum element. To fix this, we must figure out what the correct notion to replace the maximum is. This is the supremum.

\begin{definition}[supremum, infimum] \label{def:sup-inf}
    Let $A$ be a set that is bounded from below (respectively, above). A real number $s$ is the \dfn{supremum} (respectively, \dfn{infimum}) of $A$ if and only if the conditions are met.
    \begin{itemize}
        \item $s$ is an upper (respectively, lower) bound of $A$.
        \item For every upper (respectively, lower) bound $r$ of $A$, $s\leq r$ (respectively, $s\geq r$).
    \end{itemize}
    Because of these properties, $s$ is often called the least upper bound (respectively, greatest lower bound) of $A$. We write $r=\sup(A)$ (respectively, $r=\inf(A)$). 
\end{definition}
Because we wrote $\sup(A) and \inf(A)$, we should prove the following uniqueness result to justify our notation.

\begin{proposition} \label{prop:sup-uniq}
    If a subset $A\subseteq\RR$ has a supremum (respectively, infimum), it is unique.
\end{proposition}
\begin{proof}
    We prove the uniqueness of the supremum, for the proof for infimum is nearly identical and thus relegated to \Cref{exe:inf-uniq}. Suppose $s_1$ and $s_2$ are supremums of $A$. On one hand, because $s_2$ is an upper bound of $A$, we must have $s_2\leq s_1$ by the definition of $s_2$. On the other hand, because $s_1$ is an upper bound of $A$, we must have $s_1\leq s_2$. Thus, $s_1=s_2$. 
\end{proof}
\begin{exercise} \label{exe:inf-uniq}
    Complete the proof of \Cref{prop:sup-uniq} by showing that the infimum $\inf A$ is unique when it exists.
\end{exercise}

\begin{exercise}
\label{ex:sup-extend-max}
    Let $F$ be a finite set of real numbers. Show that $\sup(F)=\max(F)$ and $\inf(F) = \min(F)$.
\end{exercise}
Thus far we've shown that if the supremum of a set exists, it is unique, but how do we know such a supremum exists in the first place? If $A$ is not bounded from above, then clearly $\sup(A)$ does not exist because there is no upper bound for $A$ in the first place! But, in fact, the converse is also true.
\begin{proposition}
\label{prop:sup-prop}
    Let $A$ be a subset of real numbers bounded from above (respectively, below). Then $\sup(A)$ (respectively, $\inf(A)$) exists.
\end{proposition}
This is a key property of the real numbers, called the ``least upper bound property'' or ``supremum property.'' We won't prove it in this class since that requires us to rigorously construct the real numbers; in fact, many discussions of the real number take \Cref{prop:sup-prop} as a defining property of the real numbers! Although this property may seem intuitively true, it is not true for other ordered sets of numbers, as the next example shows.
\begin{example}
    Consider the subset of rational numbers $A\subseteq \QQ$ given by $\left\{x : x^2 < 2\right\}$. Then $\sup(A)=\sqrt{2}\not\in \QQ$. Thus, even though $A$ is a subset of rational numbers that is bounded from above, it does not have a supremum in $\QQ$, and therefore $\QQ$ does not have the supremum property.
\end{example}
It is in this sense that the real numbers ``fill in the gaps"
of the rational numbers.

\subsection{Convegence via Supremums and Infimums}
We now manifest the promise we made at the beginning of the previous subsection. With \Cref{prop:sup-prop}, we can now prove the following theorem.
\begin{theorem} \label{thm:mono-inc-converges}
    Every monotonic sequence of real numbers converges. Explicitly, if $\{x_n\}$ is monotonically increasing and bounded from above, then $\lim_{n\to \infty}x_n = \sup\{x_n : n\in \NN\}$. Similarly, if $\{x_n\}$ is monotonically decreasing and bounded from below, then $\lim_{n\to \infty}x_n = \inf\{x_n : n\in \NN\}$.
\end{theorem}
\begin{proof}
    We'll prove this for increasing sequences, as the other case is nearly identical and thus relegated to \Cref{exe:mono-dec-converges}.
    
    Set $x\coloneqq\sup\{x_n : n\in \NN\}$, which is justified by the supremum property of $\RR$, and let $\varepsilon>0$. Because $x_n\leq x$ for all $n\in \NN$, it suffices to show that there is $N$ such that $x_n > x-\varepsilon$ whenever $n\geq N$. The main claim is to show that there is some $N$ for which $x_N>x-\varepsilon$.
    
    Indeed, for the sake of contradiction, suppose that there is no $N$ such that $x_N > x-\varepsilon$. Then $x-\varepsilon \geq x_n$ for all $n$, so $x-\varepsilon$ is an upper bound of $\{x_n : n\in \NN\}$! But then $x=\sup\{x_n : n\in \NN\} \leq x-\varepsilon$, which is not possible because $\varepsilon>0$. Thus, there must be some $N$ for which $x_N\geq x - \varepsilon$. This completes the proof of the claim.
    
    We are now ready to finish the proof. Because $\{x_n\}$ is monotonically increasing,
    \[x-\varepsilon < x_N \leq x_n \leq x\]
    for all $n\geq N$, which proves that $\lim_{n\to \infty}x_n =x$.
\end{proof}
\begin{exercise} \label{exe:mono-dec-converges}
    Complete the proof of \Cref{thm:mono-inc-converges} by proving the last sentence.
\end{exercise}
This theorem allows us to rigorously find the limit of certain sequences without using an $\varepsilon$-type of argument.

\begin{example}
    Let $x\in\RR$ such that $0<x<1$, and consider the sequence $\{x^n\}$. We show $\lim x^n=0$.
\end{example}
\begin{proof}
    Note $\{x^n\}$ is a decreasing sequence (because $x<1$ implies $x^{n+1}<x^n$) that is bounded below by $0$ (because $x>0$), and therefore it converges to some $L\in \RR$ by \Cref{thm:mono-inc-converges}. Notice that the sequence $\{x^{n+1}\}$ has the same limit, and equals $\{x(x^{n})\}$, so by \Cref{prop:conv-is-linear} we have
    \begin{flalign*}
        L &= \lim_{n\to \infty}x^{n+1}
        = x\lim_{n\to \infty}x^{n}
        = xL.
    \end{flalign*}
    Because $x\neq 1$, we conclude $\lim_{n\to\infty}x^n=L=0$.
\end{proof}

\begin{example}
    Let $x_1=2$ and $x_{n+1} = \frac{x_n}{2}+\frac{1}{x_n}$ for $n\geq 1$. We show $\lim x_n=\sqrt2$.
\end{example}
\begin{proof}
    This is actually the sequence generated by using Newton's method to approximate $\sqrt{2}$. Notice that $\sqrt{2} \leq x_1\leq 2$, and if $\sqrt{2}\leq x_n\leq 2$ then
    \[ x_{n+1} = \frac{x_n}{2}  +\frac{1}{x_n} \leq 2, \]
    and
    \[ x_{n+1}^2 = \left(\frac{x_n}{2}\right)^2+1+\left(\frac{1}{x_n}\right)^2  > \frac{3}{2}+\frac{1}{x_n^2} > \sqrt{2}.\]
    Thus, by induction, $\sqrt{2}\leq x_n\leq 2$ for all $n$. Further, 
    \[ 
        x_{n}-x_{n+1} = \frac{x_n}{2} - \frac{1}{x_n} > \frac{\sqrt{2}}{2} - \frac{1}{2} > 0,
    \]
    so $\{x_n\}$ is decreasing. Therefore $\{x_n\}$ converges to the real number $L\coloneqq\inf\{x_n :n\in \NN\}$ by \Cref{thm:mono-inc-converges}. Notice that $\lim\{x_{n+1}\}=\lim\{x_n\}$, so by \Cref{prop:conv-is-linear} we have
    \begin{flalign*}
        L &= \lim_{n\to\infty} x_{n+1} \\
        &= \lim_{n\to\infty}\left(\frac{x_n}{2}+\frac{1}{x_n}\right) \\
        &= \lim_{n\to\infty}\frac{x_n}{2}+\lim_{n\to\infty}\frac{1}{x_n} \\
        &= \frac{L}{2} + \frac{1}{L}.
    \end{flalign*}
    Clearing fractions, we see that $2L^2=L^2+2$, or $L^2=2$. Because $x_n\geq \sqrt{2}$ for all $n$, and $L=\inf\{x_n :n\in \NN\}$ is a number satisfying the relation $L^2=2$, we see $L>0$ and hence $L=\sqrt{2}$.
\end{proof}
While we're here, we acknowledge that \Cref{thm:mono-inc-converges} provides for us the following nice result.
\begin{theorem}[Bolanzo--Weierstrass] \label{thm:bw}
    Let $\{x_n\}$ be a bounded sequence of real numbers. Then $\{x_n\}$ has a convergent subsequence $\{x_{n_k}\}$, where $\{n_k\}_{k\in\NN}$ is a strictly increasing sequence of natural numbers.
\end{theorem}
\begin{proof}
    The idea is to manually build a monotone subsequence of $\{x_n\}$. If $\{x_n\}$ has an infinite monotonically increasing subsequence $\{x_{n_k}\}$, then we note that $\{x_{n_k}\}_{k\in\NN}$ is a monotonically increasing subsequence whcih is bounded from above because $\{x_n\}$ is bounded from above, so $\{x_{n_k}\}$ converges by \Cref{thm:mono-inc-converges}.

    Thus, we may assume that $\{x_n\}$ has no infinite monotonically increasing subsequence $\{x_{n_k}\}$. We claim that the set $\{x_n:n\in\NN\}$ has a maximum. Indeed, supposing for contradiction that $\{x_n:n\in\NN\}$ has no maxmium, then any $x_n$ has some $x_m$ such that $x_m>x_n$. Iteratively finding $x_{n_1}$ bigger than $x_1$, then $x_{n_2}$ bigger than $x_{n_1}$, and so on, allows us to construct an infinite monotonically increasing sequence
    \[x_1<x_{n_1}<x_{n_2}<\cdots.\]
    The set $\{n_1,n_2,n_3,\ldots\}$ is infinite, so may extract from this an increasing subsequence; rigorously, starting with $n_1$, there are only finitely many numbers positive integers less than $n_1$, so eventually we may find some $n_2'$ among the elements of $\{n_2,n_3,\ldots\}$ greater than $n_1$. Then we may find some $n_3'$ after $n_2'$ and continue the process. In total, we will have produced an infinite monotonically increasing subsequence of $\{x_n\}$, which contradicts our hypothesis on $\{x_n\}$.

    Now, we have shown that $\{x_n\}$ has a maximum, which we call $x_{n_1}$. Because
    \[\{x_n:n>n_1\}\]
    is also a bounded sequence of real numbers with no infinite monotonically increasing subsequence, the argument of the previous paragraph implies that the above set also has a maximum, which we call $x_{n_2}$. By construction $x_{n_1}\ge x_{n_2}$. Continuing this process inductively produces an infinite monotonically decreasing subsequence
    \[x_{n_1}\ge x_{n_2}\ge x_{n_3}\ge\cdots,\]
    which we see must be a convergent subsequence by \Cref{thm:mono-inc-converges}.
\end{proof}

\subsection{The Extreme Value Theorem}
As a capstone to our discussion of real analysis, we will prove the Extreme value theorem, which establishes that a continuous function on a closed interval has a maximum and a minimum. This is a rather cornerstone result in calculus, where one is often interested in maximizing or minimizing the value of some fairly smooth function.

Our approach will be to first show that continuous functions on closed intervals are bounded, and then second we will go back and show that they achieve their maximums and minimums. As such, here is our boundedness result.
\begin{proposition} \label{prop:bounded-func}
    Let $a<b$ be real numbers, and let $f\colon[a,b]\to\RR$ be a continuous function. Then the set $\{f(x):x\in[a,b]\}$ is bounded. In other words, there is $M>0$ such that
    \[-M<f(x)<M\]
    for all $x\in[a,b]$.
\end{proposition}
\begin{proof}
    We will show that there is some $M>0$ such that $f(x)<M$ for all $x\in[a,b]$. Then because the function $-f$ is also a continuous function $[a,b]\to\RR$, it will follow that there is some $M'>0$ such that $-f(x)<M'$ for all $x\in[a,b]$. Then we may combine these two inequalities to achieve
    \[-\max\{M,M'\}<f(x)<\max\{M,M'\}\]
    for all $x\in[a,b]$, thus proving the result.

    Thus, it remains to show that the set $\{f(x):x\in[a,b]\}$ is bounded from above. Well, suppose for the sake of contradiction that there is no upper bound. Then for any $r>0$, there is some $x_r\in[a,b]$ such that $f(x_r)>r$. The key point, now, is that the sequence $\{x_n\}_{n\in\NN}$ is contained in $[a,b]$ and hence bounded, so \Cref{thm:bw} promises us a convergent subsequence $\{x_{n_k}\}$. For brevity, we set $x\coloneqq\lim x_{n_k}$. Because $a\le x_{n_k}\le b$ for each $k$, we note that $a\le x\le b$ by \Cref{prop:bound-seq-is-bound-limit}, so $x\in[a,b]$.

    From here, we note that \Cref{prop:seqs-and-limits} combined with the continuity of $f$ implies that
    \[f(x)=\lim_{k\to\infty}f(x_{n_k}),\]
    so in particular the sequence $\{f(x_{n_k})\}$ converges. However, $f(x_{n_k})>n_k$ for all $k$, so the sequence $\{f(x_{n_k})\}$ fails to even be bounded and thus cannot converge by \Cref{prop:conv-impy-bded}. This is the desired contradiction.
\end{proof}
We are now ready to prove the Extreme value theorem.
\begin{theorem}[extreme value]
    Let $a<b$ be real numbers, and let $f\colon[a,b]\to\RR$ be a continuous function. Then the set $\{f(x):x\in[a,b]\}$ has a minimum and a maximum. In other words, there are real numbers $x_1,x_2\in[a,b]$ such that $f(x_1)\le f(x)\le f(x_2)$ for all $x\in[a,b]$.
\end{theorem}
\begin{proof}
    We will show that there is a real number $y\in[a,b]$ such that $f(x)\le f(y)$ for all $x\in[a,b]$. For the other inequality, we note that applying the previous sentence to the continuous function $-f$ yields a real number $y'\in[a,b]$ such that $-f(x)\le-f(y')$ for all $x\in[a,b]$, which means $f(y')\le f(x)$ for all $x\in[a,b]$.

    Thus, we will content ourselves with finding $y\in[a,b]$ such that $f(x)\le f(y)$ for all $x\in[a,b]$. The idea is to construct a sequence $\{y_n\}$ such that $\{f(y_n)\}$ approaches the desired maximum value. Quickly, we note that the maximum value is
    \[M\coloneqq\sup\{f(x):x\in[a,b]\},\]
    a value which exists because $\{f(x):x\in[a,b]\}$ is bounded above by \Cref{prop:bounded-func}. Now, for any $\varepsilon>0$, we know that $M-\varepsilon$ is less than the supremum and therefore cannot be an upper bound for $\{f(x):x\in[a,b]\}$; thus, there is some $x_\varepsilon\in[a,b]$ such that $f(x_\varepsilon)>M-\varepsilon$.

    As such, the desired sequence is $\{x_{1/n}\}$. This is a sequence of real numbers in the bounded set $[a,b]$, so \Cref{thm:bw} promises us a convergent subsequence $\{x_{1/n_k}\}$. Let $y$ denote $\lim x_{1/n_k}$, and we will show $f(y)=M$. Quickly, note that we may plug in $y$ to $f$: because $a\le x_{1/n_k}\le b$ for each $k$, we see $a\le y\le b$ by \Cref{prop:bound-seq-is-bound-limit}, so $y\in[a,b]$.

    We now show $f(y)=M$, which will complete the proof. By \Cref{prop:seqs-and-limits} combined with the continuity of $f$, we see
    \[f(y)=\lim_{k\to\infty}f(x_{1/n_k}).\]
    Thus, we want to show that the right-hand limit is $M$. For this, we acknowledge that any $k$ has
    \[M-\frac1k\le M-\frac1{n_k}\le f(x_{1/n_k})\le M,\]
    so $\left|f(x_{1/n_k})-M\right|<1/k$. As such, fix any $\varepsilon>0$ and set $N\coloneqq1/\varepsilon$. Then any $k>N$ has
    \[\left|f(x_{1/n_k})-M\right|\le\frac1k<\frac1N<\varepsilon,\]
    completing the proof.
\end{proof}
%TODO introduce this 
%In other words, a continuous function on a closed interval is bounded and achieves its minimum and maximum. 
% \begin{theorem}
%     Let $f\colon[a,b]\to \RR$ be continuous. Then there is $M,N>0$ such that 
%     \[ -N\leq f(x) \leq M\]
%     and $x_1, x_2\in [a,b]$ such that $f(x_1)=-N$ and $f(x_2)=M$.
% \end{theorem}

\subsection{Problems}
Most of these statements are true with "infimum" and "bounded below" replaced appropriately. 

\begin{homework}
    Let $A\subseteq B\subseteq \RR$ be subsets.
    \begin{listalph}
        \item If $B$ is bounded from above, show that $\sup(A)\leq \sup(B)$.
        \item If $A$ is bounded from below, show that $\inf(A) \geq \inf(B)$. 
    \end{listalph}
\end{homework}

\begin{homework}
    For subsets $A,B\subseteq \RR$ which are bounded from above, let $A+B\coloneqq\{a+b: a\in A, b\in B\}$. 
    \begin{listalph}
        \item Show that $A+B$ is bounded from above. In fact, show that $\sup(A)+\sup(B)$ is an upper bound for $A+B$, so $\sup(A+B)\le\sup(A)+\sup(B)$.
        \item For any $a\in A$, show that $\sup(A+B)-a$ is an upper bound for $B$. Conclude that $\sup(B)\le\sup(A+B)-a$ and thus that $\sup(A)+\sup(B)\le\sup(A+B)$.
    \end{listalph}    
\end{homework}

\begin{homework} 
    For $t\in \RR$ and $A\subseteq \RR$, let $A+t\coloneqq\{a+t : a\in A\}$. Show that $A$ is bounded from above if and only if $A+t$ is bounded from above, and 
    \[ \sup(A+t) = \sup(A)+t.\]
\end{homework}

\begin{homework}
    For $x>0$ and $A\subseteq \RR$, let $xA \coloneqq \{xa : a\in A\}$.
    \begin{listalph}
         \item Show that $A$ is bounded from above if and only if $xA$ is bounded from.
         \item Show $\sup(xA)=x\sup(A)$.
         \item Show that $A$ is bounded from above if and only if $xA$ is bounded from below.
         \item Show $\inf(xA)=x\inf(A)$.
    \end{listalph}
\end{homework}

\begin{homework}
    Let $A\subseteq \RR$ be a subset bounded from above. Show that there is a sequence of points in $A$ converging to $\sup(A)$. 
\end{homework}

\begin{homework}
    Let $a > 1$, and consider the sequence $\{x_n\}$, defined recursively by choosing some $x_1>\sqrt a$ and then setting $x_{n+1} \coloneqq x_n + \frac{a-x_n^2}{1+x_n}$ for each $n\ge1$.
    \begin{enumerate}[label=(\alph*)]
        \item Prove that $x_1 > x_3 > x_5 > \ldots$.
        \item Prove that $x_2 < x_4 < x_6 < \ldots$.
        \item Prove that $\lim_{x \to \infty} x_n = \sqrt{a}$.
    \end{enumerate}
\end{homework}





\end{document}