%Read the slack post (link below) regarding syntax and formatting before you start writing lecture notes.
% Post: https://musa-2021.slack.com/archives/C01DGR645SL/p1609187728029500

\documentclass[../notes.tex]{subfiles}
\begin{document}

\stepcounter{week}
\section{Week \theweek: Sets and Set Operations}
To begin our foray into set theory, we discuss the definition of a set and provide some way to produce new sets from old ones. Throughout, we will motivate our discussion with many examples.

\subsection{Sets}
We begin with the definition of a set.
\begin{definition}[set, element]
    A \dfn{set} $X$ is a collection of objects. An object $x$ in a set $X$ is called an \dfn{element} or \emph{member}. If $x$ is an element of $X$, then we write $x \in X$. Similarly, if $x$ is not an element of $X$, then we write $x \notin X$. Two sets are equal if and only if they have the same elements.
\end{definition}
\begin{example}[empty set]
    There is a set, denoted $\emp$, which contains no elements. We call $\emp$ the \dfn{empty set}.
\end{example}
\begin{exercise}
    Explain why it's true that every element of $\varnothing$ is even. Is it also true that every element of $\varnothing$ is odd? 
\end{exercise}
Sets are defined by the elements they contain. In particular, we will say that two sets $X$ and $Y$ are equal if and only if they contain exactly the same elements. For example, note that $X = \{1, 1, 1, 1\}$ is the same set as $Y = \{1\}$. After all, $1 \in X$, and $1$ is the only number with this property. So sets don't recognize multiple ``copies" of their elements. Sets also do not respect order; for example, $\{1, 2, 3\} = \{3, 2, 1\}$.

We will also often want to talk about a set contained within some other, larger set.
\begin{definition}[subset]
    Let $X$ and $Y$ be sets. If $y\in Y$ implies $y\in X$ for all $y$, then we say that $Y$ is a \dfn{subset} of $X$, and we write $Y \subseteq X.$ If also $Y \neq X$, we write $Y \subsetneq X,$ and we say that $Y$ is a \dfn{proper subset} of $X$.
\end{definition}
In other words, $X\subseteq Y$ means that all the elements of $X$ live among the elements of $Y$.
\begin{warn}
    Some authors will use $Y \subset X$ to mean either that $Y$ is a subset or a proper subset of $X$! While both conventions are acceptable, it's best to choose one of the two and be as consistent with this choice as possible in your writing; to avoid ambiguity, it also helps to explicitly state when a subset is proper.
\end{warn}
For example, Barack Obama (let's denote him $O$) is an element of the set $P$ of all presidents of the United States, so we can write $O\in P$. To write down all the elements of $P$, we can say
\[P = \{\text{Joe Biden}, \text{Donald Trump}, \text{Barack Obama}, \text{George W. Bush}, \dots\}.\]
If $Q$ denotes the set of all world leaders, then $P \subseteq Q$. For example, $O \in Q$. Is $P \in Q$? No, because the set of all presidents is not a world leader.
\begin{exercise}
    Let $X=\{1,2,3\}$ and $Y=\{1,2,2,3,3,3\}$. Check that $X\subseteq Y$ and $Y\subseteq X$.
\end{exercise}
\begin{exercise}
    Fix sets $X$ and $Y$. If $X\subseteq Y$, must we have $X=Y$? If yes, provide a brief explanation with words. If no, find two sets $X$ and $Y$ with $X\subseteq Y$ while $X\ne Y$.
\end{exercise}
Next, let's define some special sets, written in blackboard font to emphasize their importance.
\begin{definition}
    The following sets will be used throughout your mathematical career.
    \begin{itemize}
        \item $\NN$ is the set of all natural numbers: $\NN \coloneqq \{0, 1, 2, \dots\}$.
        \item $\ZZ$ is the set of all integers: $\ZZ \coloneqq \NN \cup \{-1, -2, \dots\}$.
        \item $\QQ$ is the set of all rational numbers.
        \item $\RR$ is the set of all real numbers.
        \item $\CC$ is the set of all complex numbers.
    \end{itemize}
\end{definition}
\begin{warn}
    Some authors use $\NN$ to refer to the set of positive integers $\{1,2,3,\ldots\}$. We will use the notation $\ZZ^+$ to refer to this set.
\end{warn}
\begin{exercise}
    Check that $\NN \subsetneq \ZZ \subsetneq \QQ \subsetneq \RR \subsetneq \CC$.
\end{exercise}
Thus far our sets have mostly contained numbers. However, sets can contain all sorts of things.
\begin{example}
    Sets can also contain other sets! For example, $\{1\}$ and $\{1,2\}$ are (distinct) sets, and
    \[\{\{1\},\{1,2\}\}\]
    is a set distinct from both $\{1\}$ and $\{1,2\}$. Importantly, $1$ is not an element of $\{\{1\},\{1,2\}\}$ even though $\{1\}$ is!
\end{example}

\subsection{Set-Builder Notation}
In this section (and indeed, in the remainder of this course), we will use ``set-builder'' notation to write down sets. Before writing down an abstract definition using set-builder notation, we explain how it works, which is best seen by example.
\begin{example}
    Let's explain the notation
    \[E\coloneqq\{2n:n\in\ZZ\}.\]
    In other words, we are letting $E$ be the set $\{2n:n\in\ZZ\}$. Reading from left to right, the ``$2n$'' means that the set $E$ consists of elements of the form $2n$. Of course, the previous sentence does not make any sense because we have not explained what $n$ is! The right of the colon explains the context: we are considered $n\in\ZZ$. Thus, $E$ consists of elements of the form $2n$ where $n$ is an integer. Explicitly,
    \[E=\{\ldots,2\cdot-2,2\cdot-1,2\cdot0,2\cdot1,2\cdot2,\ldots\}=\{\ldots,-4,-2,0,2,4,\ldots\}.\]
    These are exactly the even integers!
\end{example}
\begin{example}
    The set $S\coloneqq\left\{n^2:n\in\ZZ\right\}$ consists of elements of the form $n^2$ where $n\in\ZZ$, so
    \[S=\left\{\ldots,(-2)^2,(-1)^2,0^2,1^2,2^2,\ldots\right\}.\]
    This is exactly the set of square integers. 
\end{example}
\begin{example}
    The set $E\coloneqq\{2n:n\in\ZZ\text{ and }n>0\}$ looks more complicated, but this is the same idea. We are still considering elements of the form $2n$, but now the context is more complicated: we are considering $n$ such that $n\in\ZZ$ and $n>0$. In other words, we are considering $n$ such that $n$ is a positive integer, so
    \[S=\{2\cdot1,2\cdot2,\ldots\}=\{2,4,\ldots\}.\]
    This is exactly the set of positive even integers.
\end{example}
To summarize, read set-builder notation as
\[\{\text{element}:\text{context}\}.\]

\subsection{Operations on Two Sets}
The simplest examples of operations on sets are operations which take in two sets and spit out a third set. Here are the main examples.
\begin{definition}[union, intersection, product]
    Let $X$ and $Y$ be sets.
    \begin{itemize}
        \item The \dfn{union} of $X$ and $Y$, written $X \cup Y$, is the set consisting of elements in $X$ or $Y$:
        \[X \cup Y \coloneqq \{z: z \in X \text{ or } z \in Y\}.\]
        \item The \dfn{intersection} of $X$ and $Y$, written $X \cap Y$, is the set consisting of elements in $X$ and $Y$:
        \[X \cap Y \coloneqq \{z: z \in X \text{ and } z \in Y\}.\]
        \item The \dfn{product} of $X$ and $Y$, written $X \times Y$, is the set of all ordered pairs of elements in $X$ and in $Y$:
        \[X \times Y \coloneqq \{(x, y): x \in X \text { and } y \in Y\}.\]
    \end{itemize}
\end{definition}
\begin{remark}
    An ordered pair is not the same thing as a set with two elements: ordered pairs allows for repetition, and order matters. Explicitly, $\{1,2\}=\{2,1\}$, but $(1,2)\ne(2,1)$.
\end{remark}
Let's do some example computations with these.
\begin{example}
    Let $X=\{1,2\}$ and $Y=\{1,2,3\}$. Then $X\cup Y=\{1,2,3\}$, and $X\cap Y=\{1,2\}$, and $X\times Y=\{(1,1),(1,2),(1,3),(2,1),(2,2),(2,3)\}$. To help us visualize the product $X\times Y$, we have the following table.
    \[\begin{array}{c|ccc}
          &     1 &     2 &     3 \\\hline
        1 & (1,1) & (1,2) & (1,3) \\
        2 & (2,1) & (2,2) & (2,3)
    \end{array}\]
\end{example}
\begin{example}
    Let $X=\{1,2\}$ and $Y=\{2,3,4\}$. Then $X\cup Y=\{1,2,3,4\}$, and $X\cap Y=\{2\}$, and $X\times Y=\{(1,2),(1,3),(1,4),(2,2),(2,3),(2,4)\}$.
\end{example}
\begin{exercise}
    Let $X=\{1,2,3\}$ and $Y=\{4,5,6\}$. Compute $X\cup Y$, and $X\cap Y$, and $X\times Y$.
\end{exercise}
\begin{example}
    Let $X=\{1,2,3\}$. Then $X\cup\emp=\{1,2,3\}$ and $X\cap\emp=\emp$. Though it looks weird, $X\times\emp=\emp$. Indeed, any ordered pair $(a,b)\in X\times\emp$ would have to have $b\in\emp$, which doesn't make any sense! Thus, $X\times\emp$ must be empty.
\end{example}
In the above examples, it looks like $X\cap Y\subseteq X$ always. This turns out to always be true. In mathematics, when we want to show that a statement is true, we provide a ``proof,'' which is basically an explanation. We'll introduce proofs to the course more formally later on, but it is good to begin our exposure to them now.
\begin{proposition} \label{prop:intersection-subset}
    Fix sets $X$ and $Y$. Then $X\cap Y\subseteq X$.
\end{proposition}
\begin{proof}
    To write this proof, we must write an explanation which works for any two sets $X$ and $Y$. Well, to show that $X\cap Y\subseteq X$, we must show that any element of $X\cap Y$ is an element of $X$.

    To proceed, it is helpful our elements names. Thus, let $z$ be any element of $X\cap Y$, and we want to show that $z$ is actually an element of $X$. Well, by definition of $X\cap Y$, we know that $z\in X$ and $z\in Y$, so we indeed see that $z\in X$. This completes the proof.
\end{proof}
Try providing explanations for the following exercises, akin to the above ``proof.''
\begin{exercise}
    Fix any set $X$. Explain, like above, why $X\subseteq X$.
\end{exercise}
\begin{exercise}
    Fix sets $X$ and $Y$. Explain, like above, why $X\cap Y\subseteq Y$.
\end{exercise}
\begin{exercise}
    Fix sets $X$ and $Y$. Explain, like above, why $X\subseteq X\cup Y$.
\end{exercise}
For something a little harder, try out the following exercise. If you get stuck, try plugging in specific sets for $A$, $B$, $C$, and $D$ to see what happens!
\begin{exercise}
    Suppose $A$, $B$, $C$, and $D$ are sets.
    \begin{listalph}
        \item Explain why $(A\times B)\cup(C\times D)\subseteq(A\cup C)\times(B\cup D)$.
        \item Find sets $A$, $B$, $C$, and $D$ such that $(A\times B)\cup(C\times D)\ne(A\cup C)\times(B\cup D)$.
    \end{listalph}
\end{exercise}
To close out this subsection, let's give a few examples of theorems about sets.
\begin{exercise}
    Let $X\coloneqq\{1,2,3\}$. Verify that $X\subseteq X$.
\end{exercise}
\begin{proposition}
    Let $X$ be any set. Then $X\subseteq X$.
\end{proposition}
\begin{proof}
    As before, writing a proof amounts to giving an explanation as we did in \Cref{prop:intersection-subset}. Unravelling the definition, to show that $X\subseteq X$, we must show that any element of $X$ is an element of $X$. But this is simply true, so we are done!
\end{proof}
\begin{example} \label{ex:subset-but-neq}
    Consider the sets $X\coloneqq\{1,2\}$ and $Y\coloneqq\{1,2,3\}$. Then we see that $X\subseteq Y$ because the elements of $X$ are $1$ and $2$, and we can check that $1\in Y$ and $2\in Y$. However, it is not the case that $Y\subseteq X$! Indeed, we see that $3\in Y$ but $3\notin Y$, so there is an element of $Y$ which is not an element of $X$, so $Y$ is not a subset of $X$.
\end{example}
\begin{exercise} \label{exe:subset-but-neq}
    Using \Cref{ex:subset-but-neq} as a guide, find sets $X$ and $Y$ such that $Y\subseteq X$, but it is not the case that $X\subseteq Y$.
\end{exercise}
\Cref{exe:subset-but-neq} is harder than it looks! It is asking for you to provide an example. Thus, to complete \Cref{exe:subset-but-neq}, you must write down two sets $X$ and $Y$ satisfying the desired property. In previous exercises, we have provided the sets for you and asked you to discuss the properties of these sets, but now you must come up with the sets yourself!

The following proposition roughly explains what is going on above.
\begin{proposition} \label{prop:subset-reflexive}
    Let $X$ and $Y$ be sets. Suppose $X\subseteq Y$ and $Y\subseteq X$. Then $X=Y$.
\end{proposition}
\begin{proof}
    This proof is going to be longer than previous ones, so pay attention! As usual, we are trying to explain why our hypotheses that $X\subseteq Y$ and $Y\subseteq X$ will imply our conclusion $X=Y$.

    Working systematically, let's begin with our hypotheses. The hypothesis $X\subseteq Y$ tells us that any element $x$ of $X$ is also an element of $Y$. Symmetrically, the hypothesis $Y\subseteq X$ tells us that any element $y$ of $Y$ is also an element of $X$. Combining the previous two sentences, we see that the sets $X$ and $Y$ must have the same elements! It follows that $X=Y$, which is what we wanted to prove.
    % We show these one at a time.
    % \begin{enumerate}[label=(\alph*)]
    %     \item For every $x \in X$, we see $x \in X$. So $X \subseteq X$.
    %     \item Assume $X \subseteq Y$ and $Y \subseteq X$. Then $x \in X$ if and only if $x \in Y$, so $X=Y$ follows.
    %     \item Assume $X \subseteq Y$ and $Y \subseteq Z$. Let $x \in X$; we must show $x \in Z$. Because $x \in X$, it follows that $x \in Y$. Because $x \in Y$, it follows that $x \in Z$.
    %     \qedhere
    % \end{enumerate}
\end{proof}
\begin{exercise}
    Consider the sets $X\coloneqq\{1,2\}$ and $Y\coloneqq\{1,2,3\}$.
    \begin{listalph}
        \item Verify that $X\ne Y$.
        \item Verify that at least one of the statements ``$X\subseteq Y$'' or $Y\subseteq X$'' is false. Are both false?
    \end{listalph}
\end{exercise}
% \begin{remark}
%     When we study relations in \cref{sec:relations}, we will see that \Cref{thm:px-is-poset} implies that the relation $\subseteq$ is a ``partial order'' of $\mc P(X)$. One also says that $\pset(X)$ is a ``poset,'' for ``partially ordered set.''
% \end{remark}

\subsection{The Power Set}
Thus far we have discussed how to take two sets and produce a third. The following definition provides us with a way to take a single set and produce another set from it.
\begin{definition}[power set, union, intersection]
    Let $X$ be a set. The \dfn{power set} of $X$, written $\mathcal P(X)$ or $2^X$, is the set of all subsets of $X$:
    \[\mathcal P(X) \coloneqq \{Y: Y \text{ is a set and } Y \subseteq X\}.\]
    % \begin{itemize}
    %     \item If $\mc F=X$ is a ``collection'' or ``family'' of sets (i.e., a set containing sets), then the \emph{union} of all the sets in $\mathcal F$, written $\bigcup \mathcal F$ or $\bigcup_{Z\in\mc F}Z$, is
    %     \[\bigcup \mathcal F \coloneqq \{z: \text{there is a set } Z \in \mathcal F \text{ such that } z \in Z\}.\]
    %     \item If $\mc F=X$ is a collection of sets, then the \emph{intersection} of all the sets in $\mathcal F$, written $\bigcap \mathcal F$ or $\bigcap_{Z\in\mc F}Z$, is
    %     \[\bigcap \mathcal F \coloneqq \{z: z\in Z\text{ for all sets } Z \in \mathcal F\}.\]
    % \end{itemize}
\end{definition}
The most important of these is the power set $\mathcal P$, so it will be our focus in the sequel.
\begin{example}
    Let $X$ be the set $\{1,2\}$. Then the subsets of $X$ are
    \[\mc P(X)=\{\emp,\{1\},\{2\},\{1,2\}\}.\]
    % Notably,
    % \[\bigcup\mc P(X)=\emp\cup\{1\}\cup\{2\}\cup\{1,2\}=\{1,2\},\]
    % and
    % \[\bigcap\mc P(X)=\emp\cap\{1\}\cap\{2\}\cap\{1,2\}=\emp.\]
\end{example}
% \begin{example}
%     The intersection of the set $P$ of presidents of the United States and the set $R$ of royalty of the United Kingdom is empty: $P \cap R = \varnothing$. Their product $P \times R$ consists of all the different ways we could pair a president with a royal; for example, $(\text{Abraham Lincoln},\text{Queen Elizabeth I})\in P \times R$.
% \end{example}
% For example,   Note that  So this element should not be confused with the set $\{\text{Abraham Lincoln},\text{Queen Elizabeth I}\}$.
Here are a few exercises for you to try.
\begin{exercise}
    Let $X$ be the set $\{1, 2, 3\}$. Determine the following.
    \begin{enumerate}[label=(\alph*)]
        \item What is the power set $\mathcal{P}(X)$?
        \item Is $1$ an element of $\mathcal{P}(X)$? What about $\{1\}$?
        \item Is $\{2, 3\}$ a subset of $\mathcal{P}(X)$? What about $\{\{2, 3\}\}$? What about $\{\{2\}, \{3\}\}$?
        \item Is $X$ an element of $\mathcal{P}(X)$? Is $X$ a subset of $\mathcal{P}(X)$?
        \item Is $\varnothing$ an element of $\mathcal{P}(X)$? Is $\varnothing$ a subset of $\mathcal{P}(X)$?
    \end{enumerate}
\end{exercise}

\subsection{Complements}
% Any \emph{binary relation} (symbol you can write between a pair of elements of a set --- we will discuss relations in greater detail in chapter 4) that has these properties is called a \dfn{partial ordering}. So we have just proved that for any set $X$, $\subseteq$ is a partial ordering of $\pset(X)$. ()
Here is the last operation of this section, requiring two or three sets depending on viewpoint.
\begin{definition}[complement]
    Suppose $A$ and $B$ are sets, both contained in a set $X$. The \dfn{complement} of $A$ in $X$, denoted $A^c$, is the set
    \[A^c \coloneqq \{x \in X: x \notin A\}.\]
    Similarly, write $A \setminus B$ for the \d{fn} (or \dfn{set difference}) of $A$ with $B$,
    \[A \setminus B \coloneqq \{x \in A : x \notin B\}.\]
\end{definition}
\begin{remark}
    Other authors might use the notation $A-B$ for the set difference $A\setminus B$. We have chosen to the notation $A\setminus B$ to distinguish our notation for set difference for the difference between two numbers.
\end{remark}
Here are the obligatory examples following our definition.
\begin{example} \label{ex:set-diff}
    Let $X$ and $Y$ be sets.
    \begin{listalph}
        \item Let $X\coloneqq\{1,2\}$ and $Y\coloneqq\{1,2,3\}$. Then $Y\setminus X$ consists of the elements in $Y$ which are not in $X$. Checking each element of $Y$, we see that the only such element is $3$, so $Y\setminus X=\{3\}$.
        \item Let $X\coloneqq\{1,2,3\}$ and $Y\coloneqq\{1,2\}$. Then $Y\setminus X$ again consists of the elements in $Y$ which are not in $X$, but checking each element of $Y$, we see that there are no such elements! So $Y\setminus X=\emp$.
    \end{listalph}
\end{example}
An important lesson from \Cref{ex:set-diff} is that it is perfectly okay to consider set differences $Y\setminus X$ even if $X$ is not a subset of $Y$. Try out the following computations.
\begin{exercise}
    Let $X=\{1,2,3\}$ and $Y=\{1,3,5\}$. Verify that $X\setminus Y=\{2\}$ and $Y\setminus X=\{5\}$.
\end{exercise}
\begin{exercise}
    Let $X=\{1,4,9\}$ and $Y=\{3,4,5,6,7\}$. Compute $X\setminus Y$ and $Y\setminus X$.
\end{exercise}
The above computations may have convinced you have some facts you can try to prove. Here is one to start.
\begin{proposition}
    Let $X$ and $Y$ be sets. Then $Y\setminus X$ is a subset of $Y$.
\end{proposition}
\begin{proof}
    Recall that $Y\setminus X$ consists of the set of elements in $Y$ that are not elements of $X$. Thus, any element of $Y\setminus X$ is an element of $Y$. This means $Y\setminus X$ is a subset of $Y$, as desired.
\end{proof}
The following examples and exercises should motivate some more propositions.
\begin{example}
    Let $X$ and $Y$ be sets.
    \begin{listalph}
        \item Let $X\coloneqq\{1,2,3\}$ and $Y\coloneqq\{1,2\}$. Then we see $X\setminus Y=\{3\}$ and $X\setminus(X\setminus Y)=\{1,2\}$.
        \item Let $X\coloneqq\{1,2,3\}$ and $Y\coloneqq\{1,3,5\}$. Then we see $X\setminus Y=\{2\}$ and $X\setminus(X\setminus Y)=\{1,3\}$.
        \item Let $X\coloneqq\{1,2,3\}$ and $Y\coloneqq\{1,3,5\}$. Then we see $Y\setminus X=\{5\}$ and $X\setminus(X\setminus Y)=\{1,3\}$.
    \end{listalph}
\end{example}
\begin{exercise}
    Let $X\coloneqq\{1,2,3,4,5\}$ and $Y\coloneqq\{2,4,6,8,10\}$.
    \begin{listalph}
        \item Compute $X\setminus Y$ and $X\setminus(X\setminus Y)$.
        \item Compute $Y\setminus X$ and $Y\setminus(Y\setminus X)$.
        \item Compute $X\cap Y$.
    \end{listalph}
\end{exercise}
The above examples motivate the following proposition.
\begin{proposition} \label{prop:double-comp}
    Let $X$ and $Y$ be sets. Then $X\setminus(X\setminus Y)=X\cap Y$.
\end{proposition}
\begin{proof}
    This proof is going to be pretty long. Take a deep breath.

    It is entirely possible to give a proof of this result similar to the proofs we have already gave previously, but it will be organize this proof if we use \Cref{prop:subset-reflexive}. Namely, to show that $X\setminus(X\setminus Y)$ equals $X\cap Y$, we will show that each set is a subset of the other. We then know that each being a subset of the other implies that they are equal by our argument in \Cref{prop:subset-reflexive}! As such, our proof is divided into two parts.
    \begin{itemize}
        \item We show that $X\cap Y$ is a subset of $X\setminus(X\setminus Y)$. Namely, for any element $x$ of $X\cap Y$, we want to show that $x$ is an element of $X\setminus(X\setminus Y)$.

        Observe that $X\setminus(X\setminus Y)$ is our ``most complicated set,'' so we begin by unravelling with that means. To show that $x$ is an element of $X\setminus(X\setminus Y)$, we want to show that $x$ is an element of $X$ but not an element of $X\setminus Y$. Well, we know that $x\in X\cap Y$, so $x\in X$ and $x\in Y$.
        
        In particular, we know $x\in X$, so it remains to show that $x$ is not an element of $X\setminus Y$. However, $x\in Y$ as discussed, so $x$ cannot be an element of $X\setminus Y$ because $X\setminus Y$ contains the elements of $X$ which are not elements of $Y$.

        The previous two paragraphs have established that $x\in X\cap Y$ implies that $x\in X\setminus(X\setminus Y)$.
        
        \item We show that $X\setminus(X\setminus Y)$ is a subset of the set $X\cap Y$. Namely, for any element $x$ of $X\setminus(X\setminus Y)$, we want to show that $x$ is an element of $X\cap Y$.
        
        As before, we start with the complicated $X\setminus(X\setminus Y)$ piece. Note $x$ being an element $X\setminus(X\setminus Y)$ means that $x$ is an element of $X$, but $x$ is not an element of $X\setminus Y$. Now, the elements of $X\setminus Y$ are the elements of $X$ which are not elements of $Y$. Thus, if $x$ is not an element of $X\setminus Y$, then $x$ is either not an element of $X$ or is an element of $Y$. However, we already know that $x\in X$ from earlier, so we must have $x\in Y$!

        To conclude, we have seen that $x\in X$ and $x\in Y$ in the previous paragraph. By definition, we see that $x\in X\cap Y$.
    \end{itemize}
    The above two arguments complete the proof once combined with \Cref{prop:subset-reflexive}.
\end{proof}
The main attraction of this subsection is a proposition called ``de Morgan's laws.'' As usual, here is a computation to motivate us.
\begin{example}
    Let $X=\{1,2,3\}$ and $Y=\{1,3,5\}$ and $Z=\{1,2,3,4\}$.
    \begin{listalph}
        \item Compute $Z\setminus X$ and $Z\setminus Y$. Then compute $(Z\setminus X)\cup(Z\setminus Y)$.
        \item Compute $X\cap Y$. Then compute $Z\setminus(X\cap Y)$.
        \item Use (a) above to compute $(Z\setminus X)\cap(Z\setminus Y)$.
        \item Compute $X\cup Y$. Then compute $Z\setminus(X\cup Y)$.
    \end{listalph}
\end{example}
% \begin{exercise}
%     If $A$ is a subset of a set $X$, express $A^c$ as a relative complement.
% \end{exercise}
% \begin{exercise}
%     Let $A$ and $B$ be subsets of a set $X$ Prove the following.
%     \begin{enumerate}[label=(\alph*)]
%         \item $A \subseteq A \cup B$.
%         \item $A \cap B \subseteq B$.
%         \item $A \setminus B \subseteq A$.
%     \end{enumerate}
% \end{exercise}
\begin{proposition}[de Morgan's laws] \nirindex{de Morgan's laws} \label{prop:de-morgan}
    Let $X$, $Y$, and $Z$ be sets. Then\todo{Move to proofs?}
    \[Z\setminus(X\cap Y)=(Z\setminus X)\cup(Z\setminus Y).\]
\end{proposition}
\begin{proof}
    As in \Cref{prop:double-comp}, we will show the above equality of sets by showing that each set is a subset of the other and conclude using \Cref{prop:subset-reflexive}.
    \begin{itemize}
        \item We show that $(Z\setminus X)\cup(Z\setminus Y)$ is a subset of $Z\setminus(X\cap Y)$. In other words, for any $z\in(Z\setminus X)\cup(Z\setminus Y)$, we want to show that $z\in Z\setminus(X\cap Y)$. Now, $z\in (Z\setminus X)\cup(Z\setminus Y)$ leaves us with two cases: either $z\in Z\setminus X$ or $z\in Z\setminus Y$.

        In one case, suppose $z\in Z\setminus X$. We want to show that $z\in Z\setminus(X\cap Y)$, so we want $z$ to be an element of $Z$ but not an element of $X\cap Y$. Well, $z\in Z\setminus X$ means that $z\in Z$ while $z\notin X$, so because $z\notin X$, we see that $z\notin X\cap Y$ as well. Thus, $z\in Z\setminus(X\cap Y)$.

        The other case is similar. Suppose $z\in Z\setminus Y$. Then $z\in Z$ and $z\notin Y$. Because $z\notin Y$, we see that $z\notin X\cap Y$ as well. Thus, $z\in Z$ but $z\notin X\cap Y$, so we conclude $z\in Z\setminus(X\cap Y)$.

        Thus, in all cases, we conclude that $z\in Z\setminus(X\cap Y)$, which is what we wanted.
        
        \item We show that $Z\setminus(X\cap Y)$ is a subset of $(Z\setminus X)\cup(Z\setminus Y)$. In other words, for any $z\in Z\setminus(X\cap Y)$, we want to show that $z\in(Z\setminus X)\cup(Z\setminus Y)$.

        Well, we know that $z\in Z\setminus(X\cap Y)$, so $z\in Z$ but $z$ is not an element of $X\cap Y$. Now, $X\cap Y$ consists of elements which are in both $X$ and $Y$, so $z\notin X\cap Y$ means that $z$ must either not be an element of $X$ or not be an element of $Y$. In other words, $z\notin X$ or $z\notin Y$.

        Now, recall $z\in Z$ already. So $z\notin X$ means that $z\in Z\setminus X$. Similarly, $z\notin Y$ means that $z\in Z\setminus Y$. But in all cases we are able to say that $z\in(Z\setminus X)\cup(Z\setminus Y)$, which is what we wanted.
    \end{itemize}
    The above arguments complete the proof by \Cref{prop:subset-reflexive}.
    % Generally speaking, to show that two sets are equal, it suffices by \Cref{prop:subset-reflexive} to show that they are subsets of each other. We will use this approach for (b) and (c).
    % \begin{enumerate}[label=(\alph*)]
    %     \item By definition, $x \in (A^c)^c$ if and only if $x$ is not in $A^c$. But $A^c$ consists of precisely those elements of $X$ not in $A$, so $x \notin A^c$ if and only if $x \in A$. In total,
    %     \[x\in(A^c)^c\text{ if and only if }x\in A,\]
    %     so $A=(A^c)^c$ follows.
    %
    %     \item In one direction, let $x \in (A \cap B)^c$, and we show $x\in A^c\cup B^c$. Then $x \notin A \cap B$. If $x \in A$ and $x \in B$, then this is a contradiction, so $x \notin A$ or $x \notin B$. So $x \in A^c \cup B^c$. Therefore, $(A \cap B)^c \subseteq A^c \cup B^c$.
    %
    %     For the other direction, let $x \in A^c \cup B^c$, and we show $x\in(A\cap B)^c$. Well, either $x \notin A$ or $x \notin B$, so $x \notin A \cap B$, so $x \in (A \cap B)^c$. Therefore, $(A \cap B)^c \supseteq A^c \cup B^c$.
    %
    %     \item This proof is similar to (b). In one direction, let $x \in (A \cup B)^c$. Thus, $x\notin A\cup B$, so $x\notin A$ or $x\notin B$. It follows $x\in A^c\cap B^c$ and so $(A\cup B)^c\subseteq A^c\cap B^c$.
    %
    %     In the other direction, let $x\in A^c\cup B^c$. Then $x\notin A$ and $x\notin B$, so $x\notin A\cup B$. It follows $x\in(A\cup B)^c$ and so $(A^c\cup B^c)\subseteq(A\cup B)^c$.
    %     \qedhere
    % \end{enumerate}
\end{proof}
\begin{exercise}
    Imitate the proof of \Cref{prop:de-morgan} to show the following. Let $X$, $Y$, and $Z$ be sets. Then
    \[Z\setminus(X\cup Y)=(Z\setminus X)\cap(Z\setminus Y).\]
\end{exercise}

\subsection{Problems}
\begin{homework}
    Let $X$ and $Y$ be sets.
    \begin{listalph}
        \item Let $X=\{1,3,5\}$ and $Y=\{1,2,3\}$. Compute $X\cap Y$. Is $X\cap Y=X$? Is $X\subseteq Y$?
        \item Let $X=\{1,3\}$ and $Y=\{1,2,3\}$. Compute $X\cap Y$. Is $X\cap Y=X$? Is $X\subseteq Y$?
        \item Suppose $X\subseteq Y$. Explain why $X\cap Y=X$.
    \end{listalph}
\end{homework}
\begin{homework}
    Let $X$, $Y$, and $Z$ be sets. Show the following.
    \begin{enumerate}[label=(\alph*)]
        \item Let $X=\{1,2\}$ and $Y=\{2,3\}$ and $Z=\{3,4\}$. Compute $X\cap (Y\cup Z)$ and $(X\cap Y)\cup(X\cap Z)$.
        \item Let $X=\{2,4\}$ and $Y=\{2,3\}$ and $Z=\{3,4\}$. Compute $X\cap (Y\cup Z)$ and $(X\cap Y)\cup(X\cap Z)$.
        \item Let $X=\{1,5\}$ and $Y=\{2,3\}$ and $Z=\{3,4\}$. Compute $X\cap (Y\cup Z)$ and $(X\cap Y)\cup(X\cap Z)$.
        \item Explain why $X \cap (Y \cup Z) = (X \cap Y) \cup (X \cap Z)$ in general.
        \item Similarly, explain why $X \cup (Y \cap Z) = (X \cup Y) \cap (X \cup Z)$ in general.
    \end{enumerate}
    (Hint: Try drawing a Venn diagram to visualize each set!)
\end{homework}
\begin{homework}
    Let $X=\{1,2,3,4\}$ be a set.
    \begin{enumerate}[label=(\alph*)]
        \item Show that $\emp\cup A=A$ for any subset $A\subseteq X$.
        \item Show that $X\cup A=X$ for any subset $A\subseteq X$.
        \item Show that $A\setminus\emp=A$ for any subset $A\subseteq X$.
        \item Show that $A\setminus X=\emp$ for any subset $A\subseteq X$.
        \item Repeat (a)--(d) for a general set $X$.
    \end{enumerate}
\end{homework}
\begin{homework}
    Let $X$ be a set. Explain why $\bigcup \mathcal P(X) = X$ and $\bigcap \mathcal P(X) = \varnothing$.
\end{homework}
\begin{homework}
    Find sets $X$ and $Y$ such that $X$ is not a subset of $Y$ and $Y$ is not a subset of $X$.
\end{homework}

\end{document}