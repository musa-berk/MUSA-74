%Read the slack post (link below) regarding syntax and formatting before you start writing lecture notes.
% Post: https://musa-2021.slack.com/archives/C01DGR645SL/p1609187728029500



\documentclass[../notes.tex]{subfiles}
\begin{document}

\stepcounter{week}
\section{Week \theweek: Functions and Relations} \label{sec:relations}
Functions appear throughout mathematics. In linear algebra, you will see functions called ``linear transformations.'' In real analysis, you may deal with sequences, metrics, and homeomorphisms---all different types of functions. In abstract algebra, you will learn about homomorphisms, another special kind of function. We begin this section by introducing functions and some of the vocabulary associated with them.

\subsection{Functions}
We begin with a few definitions.
\begin{definition}[function]
    Let $X$ and $Y$ be sets. A \dfn{function}, \dfn{mapping}, \dfn{morphism}, or \dfn{transformation} $f\colon X \to Y$ is a ``rule'' by which each element of $X$ is assigned exactly one element of $Y$. If $f$ sends $x \in X$ to $y\in Y$, we write $f(x) = y$ or $f\colon x \mapsto y$.
\end{definition}
\begin{remark}
    The truly pedantic may prefer to think of a function as its ``graph,'' which is the set
    \[\{(x,f(x)):x\in X\}.\]
    From this perspective, a function $f$ is a subset of $X\times Y$ satisfying the following condition: for each $x\in X$, there is exactly one $y\in Y$ such that $(x,y)\in f$. We will not use this perspective going forward.
\end{remark}
\begin{example}
    Here are some functions.
    \begin{itemize}
        \item Sending natural numbers $n\in\NN$ to their square $n^2\in\NN$ defines a function $f\colon\NN\to\NN$. In other words, we may define a function $f\colon\NN\to\NN$ by $f(n)\coloneqq n^2$ for each $n\in\NN$.
        \item We can define a function $g\colon\NN\to\RR$ by $g(n)\coloneqq\sqrt n$ for each $n\in\NN$.
    \end{itemize}
\end{example}
\begin{nex}
    A function $X \to Y$ assigns exactly one element of $Y$ to each element of $X$. As such, the rule which sends $x \in \RR$ to the $y \in \RR$ such that $(x, y)$ lies on the unit circle is not a function. Indeed, if $y \neq 0$, then $(x, -y)$ also lies on the unit circle, so we are sending the single $x$-value to multiple $y$-values.
\end{nex}
A rule between sending elements of $X$ to elements of $Y$ is said to be ``well-defined'' if it defines a function. If you are asked to prove that a rule $f\colon X \to Y$ is well-defined, then you should show that for all $x, x' \in X$, then $f(x) \in Y$, and if $x = y$, then $f(x) = f(y)$.
\begin{nex}
    Consider $f\colon \QQ \to \QQ$ defined by the rule
    \[f\left(\frac{a}{b}\right) \coloneqq a + b.\]
    This $f$ is not well-defined. To see this, we must find $x, y \in \QQ$ such that $x = y$ but $f(x) \neq f(y)$. Let $x = \frac{1}{2}$ and $y = \frac{2}{4}$. Then $x = y$, but $f(x) = 3$ and $f(y) = 6$, so $f$ is not well-defined.
\end{nex}
A function $f\colon X\to Y$ helps us understand the sets $X$ and $Y$. Here are the corresponding nouns.
\begin{definition}[domain, codomain]
    Let $f\colon X\to Y$ be a function. Then $X$ is called the \dfn{domain} of $f$, and $Y$ is called the \dfn{codomain} or \dfn{target} of $f$.
\end{definition}
\begin{example}
    Consider the function $f\colon\NN\to\RR$ defined by $f(n)\coloneqq\sqrt n$. Then the domain of $f$ is $\NN$, and the codomain is $\RR$.
\end{example}
\begin{exe}
    Consider the function $f\colon\ZZ\to\NN$ defined by $f(n)\coloneqq n^2$. What are the domain and codomain of $f$?
\end{exe}
Functions relate sets together, but we might also want to relate functions together. Composition is how this is done.
\begin{definition}[composition]
    Let $X$, $Y$, and $Z$ be sets, and let $f\colon X\to Y$ and $g\colon Y\to Z$ be functions. Then we define the \dfn{composition} of $f$ and $g$, denoted $(g\circ f)$, to be the function $(g\circ f)\colon X\to Z$ given by
    \[(g\circ f)(x)\coloneqq g(f(x)).\]
\end{definition}
The visual for composition is the following picture.
% https://q.uiver.app/#q=WzAsMyxbMCwwLCJYIl0sWzEsMCwiWSJdLFsxLDEsIloiXSxbMCwxLCJmIl0sWzEsMiwiZyJdLFswLDIsImdcXGNpcmMgZiIsMl1d&macro_url=https%3A%2F%2Fraw.githubusercontent.com%2FdFoiler%2Fnotes%2Fmaster%2Fnir.tex
\[\begin{tikzcd}
	X & Y \\
	& Z
	\arrow["f", from=1-1, to=1-2]
	\arrow["g", from=1-2, to=2-2]
	\arrow["{g\circ f}"', from=1-1, to=2-2]
\end{tikzcd}\]
And here are some examples.
\begin{example}
    Define the functions $f,g,h\colon\ZZ\to\ZZ$ by $f(n)\coloneqq n-1$ and $g(n)\coloneqq 2n$ and $h(n)\coloneqq n+1$.
    \begin{listalph}
        \item We compute $f\circ(g\circ h)$. To begin, we note that $(g\circ h)(n)=g(h(n))=g(n+1)=2n+2$. Thus,
        \[(f\circ(g\circ h))=f((g\circ h)(n))=f(2n+2)=2n+1.\]
        \item We compute $(f\circ g)\circ h$. To begin, we note that $(f\circ g)(n)=f(g(n))=f(2n)=2n-1$. Thus,
        \[((f\circ g)\circ h)(n)=(f\circ g)(h(n))=2h(n)-1=2(n+1)-1=2n+1.\]
    \end{listalph}
\end{example}
\begin{exe}
    Define the functions $f,g,h\colon\ZZ\to\ZZ$ by $f(n)\coloneqq2n$ and $g(n)\coloneqq n+1$ and $h(n)\coloneqq 2n-2$. Compute $f\circ(g\circ h)$ and $(f\circ g)\circ h$.
\end{exe}
\begin{exe}
    Choose three sets $W$, $X$, $Y$, and $Z$, and choose three different functions $h\colon W\to X$ and $g\colon X\to Y$ and $f\colon Y\to Z$. Then compute the functions $f\circ(g\circ h)$ and $(f\circ g)\circ h$.
\end{exe}

\subsection{Images and Pre-Images}
The function $f$ does not necessarily ``see'' all of the domain and codomain in certain situations. We reserve the words image and pre-image for these concepts.
\begin{definition}[image, pre-image]
    Let $f\colon X\to Y$ be a function.
    \begin{itemize}
        \item Given a subset $X'\subseteq X$, we define the \dfn{image} as the set of elements
        \[f(X')\coloneqq\{f(x):x\in X'\}.\]
        \item Given a subset $Y'\subseteq Y$, we define the \dfn{pre-image} as the set of elements
        \[f^{-1}(Y')\coloneqq\{x\in X:f(x)\in Y'\}.\]
    \end{itemize}
\end{definition}
Here are some computations of the image.
\begin{example}
    Consider the function $f\colon\ZZ\to\NN$ defined by $f(n)\coloneqq n^2$.
    \begin{listalph}
        \item We see $f(\{2\})=\{f(2)\}=\{4\}$.
        \item We see $f(\{-2,2\})=\{f(-2),f(2)\}=\{4\}$.
        \item We see $f(\{-3,-2,-1,0,1,2,3\})=\{f(-3),f(-2),f(-1),f(0),f(1),f(2),f(3)\}=\{0,1,4,9\}$.
    \end{listalph}
\end{example}
\begin{example}
    Let $f\colon X\to Y$ be a function. Then $\emp$ is a subset of $X$, and we see that $f(\emp)=\emp$. Indeed, $f(\emp)$ conists of elements of the form $f(x)$ where $x\in\emp$. But there are no elements $x\in\emp$, so there are no elements of the form $f(x)$ for $x\in\emp$.
\end{example}
\begin{exe}
    Consider the function $f\colon\ZZ\to\NN$ defined by $f(n)\coloneqq n^2$. Compute $f(\{5,6,7,8\})$.
\end{exe}
\begin{exe}
    Let $f\colon X\to Y$ be a function, and let $X'\subseteq X$ be a subset.
    \begin{listalph}
        \item Convince yourself that $f(X')\subseteq Y$.
        \item Find set $X$ and $Y$ and a function $f\colon X\to Y$ and a subset $X'\subseteq Y$ such that $f(X')$ is not a subset of $X$.
    \end{listalph}
\end{exe}
And here are some computations with the pre-image.
\begin{example}
    Consider the function $f\colon\ZZ\to\NN$ defined by $f(n)\coloneqq n^2$.
    \begin{listalph}
        \item Then $f^{-1}(\{4\})$ is the set of $n\in\ZZ$ such that $f(n)=4$, or $n^2=4$. Thus, we see that $f^{-1}(\{4\})=\{2,-2\}$.
        \item Then $f^{-1}(\{3\})$ is the set of $n\in\ZZ$ such that $f(n)=3$, or $n^2=3$. However, $3$ is not the square of any integer $n$, so $f^{-1}(\{3\})=\emp$.
        \item Then $f^{-1}(\{1,2,3,4\})$ is the set of $n\in\ZZ$ such that $f(n)\in\{1,2,3,4\}$, or $n^2\in\{1,2,3,4\}$. Running through each element of $\{1,2,3,4\}$, we see $f^{-1}(\{1,2,3,4\})=\{-2,-1,1,2\}$.
    \end{listalph}
\end{example}
\begin{example}
    Let $f\colon X\to Y$ be a function. Then $\emp$ is a subset of $Y$, and $f^{-1}(\emp)=\emp$. Indeed, $f^{-1}(\emp)$ consists of elements $x\in X$ such that $f(x)\in\emp$, but it is impossible to satisfy $f(x)\in\emp$, so there are no such $x\in X$.
\end{example}
\begin{exercise}
    Let $f\colon X\to Y$ be a function, and let $Y'\subseteq Y$ be a subset. Convince yourself that $f^{-1}(Y')\subseteq X$.
\end{exercise}
One interesting interaction we can already see is how the set operations we studied previously interact with the image and pre-image.
\begin{example}
    Consider the function $f\colon\ZZ\to\NN$ defined by $f(n)\coloneqq n^2$. Let $A\coloneqq\{1,2,3\}$ and $B\coloneqq\{1,3,5\}$ be subsets of $\NN$.
    \begin{listalph}
        \item We see $f(A)=f(\{1,2,3\})=\{1,4,9\}$ and $f(B)=f(\{1,3,5\})=\{1,9,25\}$. Thus,
        \[f(A)\cup f(B)=\{1,4,9,25\}\qquad\text{and}\qquad f(A)\cap f(B)=\{1\}.\]
        \item We see $A\cup B=\{1,2,3,5\}$, so $f(A\cup B)=\{1,4,9,25\}$.
        \item We see $A\cap B=\{1\}$, so $f(A\cap B)=\{1\}$.
    \end{listalph}
\end{example}
\begin{exercise}
    Consider the function $f\colon\ZZ\to\NN$ defined by $f(n)\coloneqq n^2$.
    \begin{listalph}
        \item Find two distinct integers $m,n\in\ZZ$ such that $m\ne n$ but $f(m)=f(n)$.
        \item Compute $f(\{m\})$ and $f(\{n\})$. Then compute $f(\{m\})\cap f(\{n\})$.
        \item Compute $\{m\}\cap\{n\}$. Then compute $f(\{m\}\cap\{n\})$.
        \item Use the previous parts to find subsets $A,B\subseteq\ZZ$ such that $f(A\cap B)\ne f(A)\cap f(B)$.
    \end{listalph}
\end{exercise}
Motivated by the above example, we have the following proposition.
\begin{proposition}
    Let $f\colon X\to Y$ be a function, and let $A,B\subseteq X$ be subsets. Then $f(A\cup B)=f(A)\cup f(B)$.
\end{proposition}
\begin{proof}
    As we should be used to by now, we will use \Cref{prop:subset-reflexive}. As such, there are two parts to our proof.
    \begin{itemize}
        \item We show that $f(A\cup B)\subseteq f(A)\cup f(B)$. In other words, for any $y\in f(A\cup B)$, we want to show that $y\in f(A)$ or $y\in f(B)$.

        On one hand, $f(A\cup B)$ consists of the elements of the form $f(x)$ where $x\in A\cup B$, so we may write $y=f(x)$ where $x\in A\cup B$. On the other hand, we want to show that $y\in f(A)$ or $y\in f(B)$, so we want to show that $y=f(x)$ with either $x\in A$ or $x\in B$.

        Making the two ends meet, we note that we know $y=f(x)$ where $x\in A\cup B$. But $x\in A\cup B$ means that $x\in A$ or $x\in B$. Thus, $y=f(x)$ where either $x\in A$ or $x\in B$, so $y\in f(A)$ or $y\in f(B)$. In either case, we conclude $y\in f(A)\cup f(B)$.

        \item We show that $f(A)\cup f(B)\subseteq f(A\cup B)$. In other words, for any $y\in f(A)\cup f(B)$, we want to show that $y\in f(A\cup B)$. Note that $y\in f(A)\cup f(B)$ means that either $y\in f(A)$ or $y\in f(B)$.

        In one case, suppose that $y\in f(A)$. Then $y=f(x)$ for some $x\in A$. But $x\in A$ implies that $x\in A\cup B$, so we also get to say that $y=f(x)$ for some $x\in A\cup B$. It follows $y\in f(A\cup B)$.

        In the other case, again suppose that $y\in f(B)$. Again, $y=f(x)$ for some $x\in B$, but $x\in A\cup B$ still, so $y\in f(A\cup B)$ also. Thus, all cases allow us to conclude $y\in f(A\cup B)$.
    \end{itemize}
    The above two containments allow us to complete the proof by \Cref{prop:subset-reflexive}.
\end{proof}
\begin{exe}
    Let $f\colon X\to Y$ be a function, and let $A,B\subseteq X$ be subsets. Convince yourself that $f(A\cap B)\subseteq 
    f(A)\cap f(B)$.
\end{exe}
In the problems, you will examine similar properties with the pre-image.

\subsection{Relations and Orders}
Relations allow us to compare elements within a set.
\begin{definition}[relation]
    Let $X$ be a set, and let $n$ be a positive integer. An \emph{$n$-ary \dfn{relation}} on $X$ is a subset of $X^n$, where $X^n$ denotes the set of $n$-tuples of elements of $X$. If $R$ is an $n$-ary relation on $X$ and $(x_1, \ldots, x_n) \in X^n$, then we write $R(x_1, \ldots, x_n)$ to mean $(x_1,\ldots,x_n)\in R$. If $n=2$, we might also write $x_1Rx_2$ to mean $(x_1,x_2)\in R$.
\end{definition}
We might say ``binary'' instead of $2$-ary.
\begin{example}
    Notice that a $1$-ary relation on a set $X$ is just a subset of $X$. For example, consider the $1$-ary relation $E$ on $\NN$ where $E(n)$ means ``$n$ is even.'' Then $E$ is the set of even natural numbers. 
\end{example}
\begin{example}
    Let $R\subseteq\ZZ\times\ZZ$ be the set of all ordered pairs $(x,y)$ such that $x+y$ is even. Then $R$ is relation. For example, we have $1R21$.
\end{example}
\begin{example} \label{ex:f-gives-equiv-relation}
    Define the function $f\colon\RR\to\RR$ by $f(x)\coloneqq x^2$. Then we define $R\subseteq\RR\times\RR$ by
    \[\{(x_1,x_2)\in\RR\times\RR:f(x_1)=f(x_2)\}.\]
    Then $R$ is relation. For example, $0R0$ and $2R(-2)$.
\end{example}
Orderings provide special examples of relations.
\begin{example} \label{ex:le-on-n}
    Define $L\subseteq\NN\times\NN$ by
    \[L\coloneqq\{(a,b)\in\NN\times\NN:a\le b\}.\]
    Then $L$ is a relation. For example, $1L2$, but $(2,1)\notin L$.
    % Orderings on sets, such as the usual well-ordering $\leq$ on $\NN$ that you have seen, are binary relations. Strictly speaking, the well-ordering on $\NN$ is a set $L \subseteq \NN \times \NN$ where a pair $(a, b)$ of natural numbers is an element of $L$ if and only if $a \leq b$. Thus $(3, 4)$, $(0, 42)$, and $(0, 0)$ are elements of $L$, while $(4, 3)$ and $(42, 0)$ are not elements of $L$.
\end{example}
Let's codify what we mean by an order.
\begin{definition}[parital order]
    A \dfn{partially ordered set}, or \textit{poset}, is a set $X$ with a binary relation $\preceq$ on $X$ with the following properties.
    \begin{enumerate}[label=(\alph*)]
        \item Reflexivity: for all $x \in X$, we have $x \preceq x$.
        \item Antisymmetry: for all $x, y \in X$, if $x \preceq y$ and $y \preceq x$, then $x = y$.
        \item Transitivity: for all $x, y, z \in X$, if $x \preceq y$ and $y \preceq z$, then $z \preceq z$.
    \end{enumerate}
    In this case, we call $\preceq$ a \dfn{partial order} on $X$. %A partial order $\leq$ is another type of binary relation. Note that a partial order is not necessarily \textit{total}; there may be elements $a, b \in X$ such that neither $a \leq b$ nor $b \leq a$.
\end{definition}
\begin{exercise} \label{exe:div-on-n}
    For natural numbers $a$ and $b$, we write $a \mid b$ to mean ``$b$ is divisible by $a$.'' Note $\mid$ defines a binary relation on $\NN$.
    \begin{enumerate}[label=(\alph*)]
        \item Write down three pairs of natural numbers which are elements of the relation and three pairs of natural numbers which are not in the relation.
        \item Prove that $(\NN, {}\mid{})$ is a partially ordered set.
        \item For natural numbers $a$ and $b$, write $a\nmid b$ to mean ``$b$ is not divisible by $a$.'' It is also true that $\nmid$ is a binary relation on $\NN$. Is $\nmid$ a partial order?
    \end{enumerate}
\end{exercise}
However, partial orders are a bit weak. Continuing \Cref{exe:div-on-n}, we note that $2$ does not divide $3$, and $3$ does not divide $2$. If we want our order to be a good notion of ``size,'' then we would like this sort of thing to not happen. We want total orders.
\begin{definition}[total order]
    A \dfn{totally ordered set} is a set $X$ with a partial order $\preceq$ satisfying the following fourth property.
    \begin{enumerate}[label=(\alph*)]
        \setcounter{enumi}{3}
        \item Totality: for all $x,y\in X$, we have $x\preceq y$ or $y\preceq x$.
    \end{enumerate}
    In this case, we call $\preceq$ a \dfn{total order} on $X$.
\end{definition}
\begin{example}
    The relation $L$ of \Cref{ex:le-on-n} is a total order.
\end{example}
\begin{nex}
    The partially ordered set $(\NN,{}\mid{})$ of \Cref{exe:div-on-n} is not a total order. To see this, we note that $3,5\in\NN$ makes both $3\mid 5$ and $5\mid 3$ false, so we fail totality!
\end{nex}

\subsection{Equivalence Relations}
A common way to compare two things is to say that they are similar to each other. For sets, the way to say that two elements of a set are similar to each other is with an equivalence relation.
\begin{definition}[equivalence relation]
    Let $X$ be a set, and let $\sim$ be a binary relation on $X$. We will use the notation $x \sim y$ to mean that $\sim$ holds of the pair $(x, y)$. Say that $\sim$ is an \dfn{equivalence relation} on $X$ if all the following hold:
    \begin{enumerate}[label=(\alph*)]
        \item Reflexivity: for all $x \in X$, we have $x \sim x$.
        \item Symmetry: for all $x, y \in X$, if $x \sim y$, then $y \sim x$.
        \item Transitivity: for all $x, y, z \in X$, if $x \sim y$ and $y \sim z$, then $x \sim z$.
    \end{enumerate}
\end{definition}
In some sense, equivalence relations are supposed to generalize equalities. To see this, we show that equality is an equivalence relation, in the following sense.
\begin{example} \label{ex:eq-is-equiv}
    Let $X=\ZZ$. Define the binary relation $\sim$ on $X$ by $x\sim x$ if and only if $x=x$. Then $\sim$ is an equivalence relation.
    \begin{listalph}
        \item Reflexivity: for all $x\in X$, we have $x=x$, so $x\sim x$.
        \item Symmetry: for all $x,y\in X$, if $x\sim y$, then $x=y$, so $y=x$ too, so $y\sim x$.
        \item Transitivity: for all $x,y,z\in X$, if $x\sim y$ and $y\sim z$, then $x=y$ and $y=z$, so it follows $x=z$, so $x\sim z$.
    \end{listalph}
    Note that the above argument works for any set $X$.
\end{example}
\begin{exe}
    Show that the relation $R$ of \Cref{ex:f-gives-equiv-relation} is an equivalence relation.
\end{exe}
Here is an example that will be relevant to us later in \cref{subsec:mods}. It is somewhat more complicated than what we have done so far.
\begin{prop} \label{prop:mods-equiv-relation}
    Let $n$ be a positive integer, and let $\sim$ be the binary relation on $\ZZ$ where $x \sim y$ if and only if $x-y$ is divisible by $n$. Then $\sim$ is an equivalence relation.
\end{prop}
\begin{proof}
    To prove that $\sim$ is an equivalence relation, we must show that $\sim$ is reflexive, symmetric, and transitive.
    \begin{enumerate}[label=(\alph*)]
        \item Reflexivity: let $x \in \ZZ$. Then $x-x=0$ is divisible by $n$ because $0=0\cdot n$, so $x\sim x$.
        \item Symmetry: let $x, y \in \ZZ$, and suppose $x \sim y$. Then $x-y$ is divisible by $n$, so we can find an integer $a$ such that $x-y=an$. But then
        \[y-x=-1\cdot(x-y)=-a\cdot n,\]
        so $y-x$ is divisible by $n$, so $y\sim x$.
        \item Transitivity: let $x, y, z \in \ZZ$, and suppose that $x \sim y$ and $y \sim z$. Then $x-y$ and $y-z$ are divisible by $n$, so we can find integers $a$ and $b$ such that $x-y=an$ and $y-z=bn$. Summing, we see
        \[x-z=(x-y)+(y-z)=an+bn=(a+b)n.\]
        Thus $x-z$ is divisible by $n$, so $x\sim z$.
        \qedhere
    \end{enumerate}
\end{proof}
One can generalize \Cref{ex:eq-is-equiv} as follows. The following proposition approximately says that equivalence relations are equalities ``up to squishing by a function.''
\begin{proposition} \label{prop:function-to-equiv}
    Let $X$ be a set and $f\colon X\to Y$ be a function. Then define the relation $\sim_f$ on $X$ by $x_1\sim_f x_2$ if and only if $f(x_1)=f(x_2)$. Then $\sim_f$ is an equivalence relation.
\end{proposition}
\begin{proof}
    As before, to show that $\sim_f$ is an equivalence relation, we must show that $\sim_f$ is reflexive, symmetric, and transitive.
    \begin{enumerate}[label=(\alph*)]
        \item Reflexivity: for each $x\in X$, we note $f(x)=f(x)$, so $x\sim_fx$.
        \item Symmetry: given $x,y\in X$ such that $x\sim_fy$, we see $f(x)=f(y)$. But then $f(y)=f(x)$, so $y\sim_fx$ as well.
        \item Transitivity: suppose $x,y,z\in X$ have $x\sim_fy$ and $y\sim_fz$. Then $f(x)=f(y)$ and $f(y)=f(z)$, so $f(x)=f(z)$, meaning $x\sim_fz$.
        \qedhere
    \end{enumerate}
\end{proof}
We said that equivalence relations declare elements similar to each other, so it is useful to talk about all the elements similar to each other as one object.
\begin{definition}[equivalence class]
    Let $X$ be a set and $\sim$ be an equivalence relation on $X$. An \emph{equivalence class} is a subset $Y\subseteq X$ such that, for all $y_1, y_2 \in Y$, we have $y_1 \sim y_2$. If $x \in X$, then the set
    \[[x] \coloneqq \{y \in X : x \sim y\}\]
    is the equivalence class of $x$.
\end{definition}
\begin{remark}
    Technically, we must check that $[x]$ is in fact an equivalence class. For completeness, we do this: if $y_1,y_2\in[x]$, we must show $y_1\sim y_2$. Well, by definition of $[x]$, we see $x\sim y_1$ and $x\sim y_2$. But $\sim$ is an equivalence relation! It follows $y_1\sim x$ and $x\sim y_2$, so $y_1\sim y_2$.
\end{remark}
\begin{example}
    Let $n$ be a positive integer. Using \Cref{prop:mods-equiv-relation}, consider the equivalence relation $\sim$ on $\ZZ$ where $x \sim y$ if and only if $x-y$ is divisible by $n$. Then $[0]$ is the set of multiples of $n$. One can check that the other equivalence classes are $[1], [2], \ldots, [n - 1]$.
\end{example}
\begin{example}
    Using the relation $R$ of \Cref{ex:f-gives-equiv-relation}, we see that
    \[[1]=\left\{x\in\RR:x^2=1^2\right\}=\{\pm1\}.\]
    More generally, $[y]=\{\pm y\}$.
\end{example}
Having divided our set into equivalence classes, we now pick up the equivalence classes back again.
\begin{definition}
    Let $X$ be a set and $\sim$ an equivalence relation on $X$. Then $X / {\sim}$, usually pronounced ``$X$ mod $\sim$,'' is the set of equivalence classes of $X$ under the equivalence relation $\sim$.
\end{definition}
\begin{example}
    Let $n$ be a positive integer. Using \Cref{prop:mods-equiv-relation}, consider the equivalence relation $\sim$ on $\ZZ$ where $x \sim y$ if and only if $x-y$ is divisible by $n$. Then we saw $\ZZ /{\sim}$ is the set $\{[0], [1], [2], \dotsc, [n]\}$.
\end{example}
\begin{exercise}
    The following problem is similar to one you will likely see in Math 113. Fix a positive integer $n$, and consider again the equivalence relation $\sim$ on $\ZZ$ where $x \sim y$ if and only if $x-y$ is divisible by $n$. Recalling $\ZZ/{\sim}=\{[0], [1], [2], \dotsc, [n - 1]\}$, show that the function $f\colon (\ZZ /{\sim}) \times (\ZZ /{\sim}) \to \ZZ /{\sim}$, given by
    \[f\colon([a], [b]) \mapsto [a + b],\]
    is well-defined.
\end{exercise}
% The notion of $X/{\sim}$ allows us to build the following reversed version of \Cref{prop:function-to-equiv}. This tells us that functions and equivalence relations are inherently intertwined: any function gives an equivalence relation, and any equivalence relation gives back a function.
% \begin{exe}
%     Let $\sim$ be an equivalence relation on a set $X$. Define the function $f\colon X\to X/{\sim}$ by sending each $x\in X$ to the equivalence class $[x]\in X/{\sim}$. Show that $x\sim x'$ if and only if $x\sim_fx'$, where $\sim_f$ has been defined as in \Cref{prop:function-to-equiv}.
% \end{exe}
% This next definition gives another way to think about equivalence relations.
% \begin{definition}[partition]
%     Let $X$ be a set. A \dfn{partition} of $X$ is a family $\mathcal{F}$ of subsets of $X$ such that the following hold.
%     \begin{enumerate}[label=(\alph*)]
%         \item Distinct elements of the partition are disjoint: if $A, B \in \mathcal{F}$ and $A \neq B$, then $A \cap B = \varnothing$.
%         \item Every element of the partition is nonempty: if $A \in \mathcal{F}$, then $A \neq \varnothing$.
%         \item The union of $\mathcal{F}$ is the entire set $X$: in other words, every element of $X$ is found in some $A \in \mathcal{F}$.
%     \end{enumerate}
% \end{definition}
% \begin{exercise}
%     Let $X$ be a set, and let $\mathcal{F}$ be a partition of $X$. Define $\sim$ to be the binary relation on $X$ such that $x \sim y$ if and only if there is $A \in \mathcal{F}$ such that $x \in A$ and $y \in A$. Prove that $\sim$ is an equivalence relation.
% \end{exercise}
% \begin{exercise}
%     Let $X$ be a set, and let $\sim$ be an equivalence relation on $X$. Prove that $X /{\sim}$ is a partition of $X$.
% \end{exercise}
% Again, the previous two exercises tell us that partitions are intertwined with equivalence relations: any partition gives an equivalence relation, and any equivalence relation gives back a partition.

\subsection{Problems}
\begin{homework}
    Define $f\colon\ZZ\to\NN$ by $f(n)\coloneqq n^2$, and let $A=\{1,2,3,4,5\}$ and $B=\{1,3,5,7,9\}$.
    \begin{listalph}
        \item Compute $f^{-1}(A\cap B)$ and $f^{-1}(A)\cap f^{-1}(B)$.
        \item Compute $f^{-1}(A\cup B)$ and $f^{-1}(A)\cup f^{-1}(B)$.
    \end{listalph}
\end{homework}
\begin{homework}
    Let $f\colon X\to Y$ be a function and let $A,B\subseteq Y$ be subsets of $Y$.
    \begin{enumerate}[label=(\alph*)]
        \item Convince yourself that $f^{-1}(A\cap B)=f^{-1}(A)\cap f^{-1}(B)$.
        \item Convince yourself that $f^{-1}(A\cup B)=f^{-1}(A)\cup f^{-1}(B)$.
    \end{enumerate}
\end{homework}
\begin{homework}
    Let $S$ be a set, and recall that $\mc P(S)$ denotes the power set of $S$. For any $A \subseteq S$, define $1_A\colon S \to \{0, 1\}$ be defined by
    \[1_A(s) = \begin{cases} 1 & \text{if }s \in A, \\ 0 & \text{if }s \notin A.\end{cases}\]
    Prove that $1_A$ is well-defined. This function $1_A$ is called the ``characteristic function of $A$ in $S$.''
\end{homework}
% \begin{homework}[first isomorphism theorem of vector spaces]
%     \index{first isomorphism theorem!linear algebra}
%     For those who have taken a linear algebra course, show that if $V$ and $W$ are (say, finite-dimensional) vector spaces and $T: V \to W$ is a linear transformation, there are subspaces (not just subsets!) $V_0 \subseteq V$ and $W_0 \subseteq W$, and linear transformations (not just functions!) $\pi: V \to V_0$, $\iota: W_0 \to W$, and $\tilde T: V_0 \to W_0$ such that $\pi$ is surjective, $\iota$ is injective, and $\tilde T$ is invertible, and such that the diagram
% $$\begin{tikzcd}
%     V \arrow[r, "T"] \arrow[d, "\pi"] & W \\
%     V_0 \arrow[r, "\tilde T"] & W_0 \arrow[u, "\iota"]
%     \end{tikzcd}$$
%     commutes; in other words,
%     $$T = \iota \circ \tilde T \circ \pi.$$
%     Moreover, prove that $\dim V_0 + \dim \ker T = \dim V$ and $\dim W_0 = \rk T$. To accomplish this, you'll probably want to use rank theorem and the invertible matrix theorem.
% \end{homework}
% \begin{homework}
%     Extend \Cref{prop:factor-functions} as follows: let $T\colon V\to W$ be a linear transformation of vector spaces. Show there exists a vector space $U$ and linear transformations $\pi\colon V\to U$ and $\iota\colon U\to W$ such that $\pi$ is surjective, $\iota$ is injective, and
%     \[T=\iota\circ\pi.\]
% \end{homework}
\begin{homework}
    For each of the following rules, either prove the rule defines a function or show it is not well-defined.
    \begin{enumerate}[label=(\alph*)]
        \item $f\colon \QQ \to \QQ$ by $\frac{a}{b} \mapsto ab$
        \item $f\colon \NN \to \NN$ by $n \mapsto n + 1$.
        \item $f\colon \ZZ^+ \times \ZZ^+ \to \ZZ$ by $(a, b) \mapsto \frac{a}{b}$, where $\ZZ^+$ is the set of positive integers.
        \item $f\colon \QQ^+ \to \QQ^+$ by $\frac ab\mapsto\frac ba$, where $\QQ^+$ is the set of positive rational numbers.
        \item $f\colon X \to (X / {\sim})$ by $x \mapsto [x]$, where $\sim$ is an equivalence relation on a set $X$.
        \item $f\colon (X / {\sim}) \to X$ by $[x] \mapsto x$, where $\sim$ is an equivalence relation on a set $X$.
    \end{enumerate}
\end{homework}
\begin{homework}
    Define the function $f\colon\ZZ\to\{0,1\}$ by
    \[f(n)\coloneqq\begin{cases}
        0 & \text{if }n\text{ is even}, \\
        1 & \text{if }n\text{ is odd}.
    \end{cases}\]
    \begin{listalph}
        \item Select any two distinct even integers $a,b\in\ZZ$. Show that $f(a)=f(b)$.
        \item For any two even integers $a,b\in\ZZ$, show that $f(a)=f(b)$.
        \item Repeat (a)--(b) with the word ``even'' replaced by the word ``odd.''
    \end{listalph}
\end{homework}
\begin{homework}
    This exercise will teach you how to construct the set of integers from equivalence classes on the set of pairs of natural numbers. For $(a, b), (c, d) \in \NN \times \NN$, say $(a, b) \sim (c, d)$ if and only if $a + d = b + c$. We claim that $(\NN \times \NN) /{\sim}$ looks very much like $\ZZ$.
    \begin{enumerate}[label=(\alph*)]
        \item Prove that $\sim$ is an equivalence relation.
        \item Write $(\NN \times \NN) /{\sim}$ as $\mathcal{Z}$. Define $\alpha\colon \ZZ \mapsto \mathcal{Z}$ by
        \[z \mapsto \begin{cases} [(z, 0)] & \text{if }z \geq 0, \\ [(0, -z)] & \text{if }z < 0. \end{cases}\]
        Prove that $\alpha$ is well-defined and a bijection. Where does the inverse function send $[(a,b)]$?
        \item Define $+_{\mathcal{Z}}\colon \mathcal{Z} \times \mathcal{Z} \to \mathcal{Z}$ by
        $$[(a, b)] +_{\mathcal{Z}} [(c, d)] = [(a + c, b + d)]$$
        Show that $+_{\mathcal{Z}}$ is well-defined. In fact, show $\alpha(a) +_{\mathcal{Z}} \alpha(b) = \alpha(a + b)$ for any $a,b\in\ZZ$.
    \end{enumerate}
    % You have shown that $\mathcal{Z}$ with the addition operation $+_{\mathcal{Z}}$ is structurally very similar to $\ZZ$ with the usual addition.
\end{homework}

\end{document}